\section{Contexto social}

Las tecnologías de la información en general e Internet en particular son ya
parte integral de la sociedad. Casi todos los ámbitos de la vida,
desde las interacciones sociales hasta la búsqueda de empleo, cuentan ya con
su reflejo en las tecnologías de la información, a menudo mediante el uso de
servicios web a través de Internet.

No solo los aspectos tradicionales de la sociedad tienen su presencia en las
redes, también han surgido nuevos modelos empresariales propios de Internet que
han crecido de manera importante y se han situado a niveles comparables a los de
las empresas tradicionales. Empresas puramente digitales como Facebook o Twitter
ya cotizan en bolsa y realizan operaciones bursátiles del orden de miles de
millones de dólares~\cite{facebook-acquires-whatsapp}.

Se pone pues de manifiesto la importancia de la fiablidad de los servicios e
infraestructuras de los que dependen estos nuevos modelos de negocio. El
\textit{uptime} -- un término inglés que describe el porcentaje de tiempo que un
servicio se mantiene disponible -- debe ser siempre cercano al 100\%, dado que
en caso contrario los potenciales usuarios del servicio se encontrarán con que
no pueden acceder a él, dando lugar incluso a pérdidas económicas. Es el caso de
Amazon, que llegó a perder 4.8 millones de dólares al sufrir un fallo que dejó
inaccesible su web durante 40 minutos~\cite{amazon}.

De todo lo expuesto se extrae la necesidad de contar con sistemas para
monitorizar la disponibilidad de estos servicios y, en caso necesario, actuar de
manera que puedan subsanarse las causas de los problemas.

\section{Motivación}
\label{sec:situacion-actual}

La idea de desarrollar este proyecto surge a raíz de una necesidad
personal del desarrollador del proyecto. 

En octubre de 2013 tomé parte en un importante proceso de selección en el que
las comunicaciones se estaban llevando a cabo a través de correo electrónico. En
particular, la cuenta de correo electrónico que estaba usando
(\texttt{info@josetomastocino.com}) tenía un dominio personalizado y estaba
gestionada a través de Google Apps~\cite{google-apps}. Esto era posible gracias
a que en el dominio se dieron de alta unos registros de tipo MX y TXT que hacían
que el correo se redirigiese a los servidores de Google.

A mitad del proceso de selección, la empresa que gestionaba el dominio de mi web
personal (y, por extensión, mi correo electrónico) tuvo un problema y reemplazó
los registros DNS personalizados por unos por defecto. En particular, los
registros MX y TXT quedaron en blanco, y los registros A apuntaron a una página
de párking\footnote{Una página de \textit{párking de dominio} sirve como página
  temporal mientras un servicio web no es lanzado, de forma que el dominio no
  esté en blanco.}, de forma que los chequeos de tipo ping que tenía puestos no
dieron error, ya que efectivamente el servidor devolvía el ping, pero el
contenido de los registros no era el correcto, ni los correos se estaban
direccionando bien.

Esta situación se prolongó durante varios días, en los cuales perdí varios
mensajes importantes. Esto podría haberse detectado rápidamente si hubiera tenido
alguna clase de vigilancia que verificase que los registros DNS tenían el
contenido correcto. 

De ahí nace la necesidad de realizar un estudio de las alternativas existentes
para este tipo de vigilancia y, al concluir que no existía ninguna alternativa
libre, se decide iniciar el proyecto SiteUp para la vigilancia de servicios de
Internet.


\section{Objetivos}
\label{sec:objetivos}

A la hora de definir los objetivos de un sistema, podemos agruparlos
en dos tipos diferentes: \textbf{funcionales} y
\textbf{transversales}. Los primeros se refieren a \textit{qué} debe
hacer la aplicación que vamos a desarrollar, e inciden
directamente en la experiencia del usuario y de potenciales
desarrolladores.

Por otro lado, los objetivos transversales son aquellos invisibles al
usuario final, pero que de forma inherente actúan sobre el resultado
final de la aplicación y sobre la experiencia de desarrollo de la misma.

\subsection{Funcionales}
\begin{itemize}
\item Crear un conjunto de herramientas para la monitorización y el chequeo de
  diversos aspectos del estado de un servicio de Internet.
\item Crear una aplicación online, de acceso público, que permita la creación y
  gestión de chequeos de manera sencilla, basada internamente en las
  herramientas mencionadas en el punto anterior.
\item Habilitar a esta aplicación de un sistema de notificaciones mediante correo
  electrónico que alerte a los usuarios de posibles cambios en la disponibilidad
  de los servicios monitorizados.
\item Crear una aplicación móvil para que los usuarios tengan la opción de
  recibir notificaciones instantáneas provenientes de la aplicación web con
  información de sus chequeos.
\end{itemize}

\subsection{Transversales}
\begin{itemize}
\item Investigar y conocer los vectores de vigilancia usados habitualmente para
  monitorizar servicios de Internet.
\item Ampliar mis conocimientos sobre desarrollo web en general y las
  tecnologías de back-end en particular.
\item Adquirir soltura en el uso del lenguaje de programación Python en entornos
  web.
\item Obtener una base de conocimientos mínima sobre el desarrollo de
  aplicaciones sobre la plataforma móvil Android.
\item Utilizar un enfoque de análisis, diseño y codificación orientado
  a objetos, de una forma lo más clara y modular posible, para
  permitir ampliaciones y modificaciones sobre la aplicación por
  terceras personas.
\item Hacer uso de herramientas básicas en el desarrollo de software,
  como son los \textbf{sistemas de control de versiones} para llevar
  un control realista del desarrollo del software, así como hacer de
  las veces de sistema de copias de seguridad.
\end{itemize}

\section{Estado del arte}
\label{sec:estado-del-arte}

Se detalla a continuación el estado del arte de los puntos de vigilancia de
servicios más habituales y de los servicios web más similares a lo que se busca
construir en el proyecto.

\subsection{Métodos de monitorización}

La monitorización de servicios en general, y de servicios remotos a través de
Internet en particular, ha sido siempre un aspecto importante en el despliegue
de cualquier proyecto. Existe un gran número de vectores de vigilancia que se
han usado a lo largo de los años. A continuación se detallan los más habituales.

\subsubsection{Envío de paquetes ICMP}

\ac{ICMP} es uno de los principales protocolos de la familia TCP/IP. Es
utilizado por dispositivos de red, como los enrutadores, para enviar mensajes de
control con diversas funciones como, por ejemplo, mensajes de error informando
de que una máquina remota no está disponible.

La herramienta más conocida que trabaja sobre ICMP es la utilidad \textbf{Ping},
que envía peticiones \textit{eco} a un host remoto, y espera la
respuesta. Durante ese proceso, la herramienta mide el tiempo desde el envío a
la recepción, monitorizando las posibles pérdidas de paquetes y otros
valores. Respuestas incorrectas o inexistentes permiten al usuario tener una
idea de que la máquina remota no es accesible de forma correcta.

Un ejemplo de la salida del comando \texttt{ping} en un sistema GNU/Linux es la
siguiente:

\begin{bashcode}
$ ping -c3 uca.es
PING uca.es (150.214.80.25) 56(84) bytes of data.
64 bytes from www21.uca.es (150.214.80.25): icmp_seq=1 ttl=48 time=43.3 ms
64 bytes from www21.uca.es (150.214.80.25): icmp_seq=2 ttl=49 time=42.7 ms
64 bytes from 150.214.80.25: icmp_seq=3 ttl=49 time=41.2 ms

--- uca.es ping statistics ---
3 packets transmitted, 3 received, 0% packet loss, time 3207ms
rtt min/avg/max/mdev = 41.228/42.421/43.300/0.906 ms
\end{bashcode}

El uso del comando ping para el chequeo básico de servidores es algo habitual
desde hace más de dos décadas~\cite{salus1994a}. A pesar ello, hay muchos
servidores que tienen deshabilitada su capacidad de responder a esta clase de
paquetes, por lo que es imperativo contar con otras medidas de chequeo
complementarias.

\subsubsection{Chequeo de peticiones HTTP}
\label{subsec:chequeoHttp}

A la hora de verificar el correcto funcionamiento de un servidor web, la opción
más lógica es utilizar peticiones \ac{HTTP} contra el servidor, tal y como haría
un cliente web normal (como un navegador). El protocolo HTTP es la base de la
comunicación en la web, y tiene sistemas para verificar que el envío y recepción
de las peticiones es correcto.

Cuando un navegador u otro cliente hace una petición a un servidor web, éste
devuelve una respuesta que comienza con un \textbf{código de
  estado}, seguido de unas cabeceras, una línea en blanco y el cuerpo del
mensaje. El aspecto de las cabeceras es el siguiente

\begin{verbatim}
HTTP/1.1 200 OK
Server: nginx/1.4.1 (Ubuntu)
Date: Mon, 28 Apr 2014 17:53:26 GMT
Content-Type: text/html; charset=utf-8
Connection: keep-alive
Vary: Accept-Encoding
X-Powered-By: PHP/5.5.3-1ubuntu2.2
\end{verbatim}

En la primera línea se observa el \textbf{código de estado}, que indica el
estado de la petición. Estos códigos fueron establecidos por la \ac{IETF}, se
componen de tres dígitos y contemplan un extenso número de situaciones,
divididas en cinco familias según el primer dígito del código:

\begin{itemize}
\item \textbf{1xx, informativos}. Informan de que la petición ha sido recibida y
  que puede continuarse el proceso, por ejemplo enviando por parte del cliente
  el resto de la petición.
\item \textbf{2xx, petición correcta}. Indica que la petición fue correctamente
  recibida, entendida y aceptada. El código más conocido de esta familia es el
  \textbf{200 - OK}, la respuesta estándar para peticiones correctas.
\item \textbf{3xx, redirección}. Indica que se debe producir una redirección
  para completar la petición por parte del cliente. El código más popular en
  esta familia es el \textbf{302 - Moved temporarily}.
\item \textbf{4xx, error del cliente}. Indica que el cliente ha cometido un
  error al realizar la petición, por ejemplo pidiendo un recurso inexistente. Es
  el caso del código \textbf{404 - Not Found}.
\item \textbf{5xx, error del servidor}. Indica que aunque la petición parece
  correcta, el servidor ha sufrido un fallo al intentar responderla. El mensaje
  más habitual es el \textbf{500 - Internal Server Error}, que informa de manera
  genérica de un error interno.
\end{itemize}

Así, un punto de vigilancia habitual es comprobar que las peticiones hacia
cierta \ac{URL} devuelven el código de estado correcto. Que una web que
normalmente devuelve un código 200 de pronto devuelva el código 500 es un
indicador de que ha ocurrido un fallo y el sistema no funciona apropiadamente.

Incluso tras la verificación del correcto código de estado uno no puede estar
del todo seguro de que una web funcione correctamente. Hay un caso especial de
ataque, el \textit{defacement}~\cite{wiki:defacement}, que consiste en cambiar
de forma total o parcial el contenido de una página web, manteniendo su
accesibilidad. Así, las comprobaciones de tipo \textit{ping} o de código de
estado serán correctas, ya que el servidor seguirá en línea y respondiendo
peticiones, pero el contenido no será el adecuado.

Para estos casos es importante establecer un sistema de verificación del
contenido. La forma más sencilla y ligera es decidir una serie de palabras clave
que deberán encontrarse en el contenido. Así, el sistema de vigilancia
verificará en cada chequeo que esas palabras clave se encuentren en el cuerpo de
la respuesta. Otra opción es generar inicialmente una imagen del contenido, ya
sea haciendo una copia íntegra en la base de datos o guardando un resumen
mediante un algoritmo de \textit{hashing}, y compararla en cada chequeo. Con
esto nos aseguramos de que la totalidad del contenido coincide con lo
esperado. La contrapartida es que este sistema no puede utilizarse en páginas
dinámicas, ya que cualquier cambio en el contenido haría saltar la alerta.

\subsubsection{Chequeo de puertos}

El envío de paquetes ICMP está bien para comprobaciones genéricas, y el uso de
peticiones HTTP es el más apropiado para servidores web, pero es necesario un
término medio que sirva para vigilar otros servicios remotos de forma algo más
exhaustiva pero manteniendo la adaptabilidad.

Estos servicios remotos utilizan \textbf{puertos lógicos} a través de los cuales
los clientes pueden conectarse y obtener información. Los servidores web no son
ninguna excepción, y suelen utilizar el puerto 80 como interfaz de comunicación
con el exterior, pero hay muchos otros. Se pueden usar los números entre 0 y
65536 para identificar un puerto, y la \ac{IANA} es responsable de asignar los
números de puerto por debajo del 1024 a los servicios más habituales:

\begin{itemize}
\item 20 y 21: File Transfer Protocol (FTP)
\item 22: Secure Shell (SSH)
\item 23: Telnet remote login service
\item 25: Simple Mail Transfer Protocol (SMTP)
\item 53: Domain Name System (DNS)
\item 80: Hypertext Transfer Protocol (HTTP)
\item 110: Post Office Protocol (POP3)
\item 119: Network News Transfer Protocol (NNTP)
\item 143: Internet Message Access Protocol (IMAP)
\item 161: Simple Network Management Protocol (SNMP)
\item 194: Internet Relay Chat (IRC)
\item 443: HTTP Secure (HTTPS)
\item 465: SMTP Secure (SMTPS)
\end{itemize}

La verificación más habitual en este caso es abrir una conexión al puerto
remoto, comprobando que la conexión se realiza correctamente y no es denegada
por el servidor remoto. Opcionalmente se puede comprobar la respuesta textual
del servidor, verificando que se obtiene un mensaje de bienvenida o algo
similar, habitual en servicios como SSH o FTP. Por ejemplo, un servidor SSH
puede devolver como línea inicial el siguiente mensaje:

\begin{verbatim}
SSH-2.0-OpenSSH_6.2p2 Ubuntu-6ubuntu0.1
\end{verbatim}

Con esto nos cercioramos de que el servidor está en línea, es accesible, tiene
abierto el puerto asociado al servicio, éste acepta conexiones y responde con el
mensaje habitual.

\subsubsection{Verificación de dominios y registros DNS}

Hasta ahora, los mecanismos de monitorización que se han mencionado actúan
directamente sobre el servidor remoto que ofrece el servicio a monitorizar,
normalmente a través de una \ac{URL} o un \textit{hostname}. Desafortunadamente
existen vectores de ataque que se dirigen hacia servicios relacionados pero no
alojados o gestionados directamente por el servidor. Un ejemplo de esto es la
gestión de los registros \ac{DNS}.

En cualquier red, y en Internet en particular, los sistemas conectados se
identifican generalmente utilizando una dirección \ac{IP}, normalmente asignada
por el \ac{ISP} con el que se haya contratado el acceso o por el administrador
de la red. Dado que recordar una dirección IP es algo difícil, surgen los
\textbf{nombres de dominio}, cuyo principal propósito es el de ofrecer términos
memorizables y fáciles de encontrar. Así, a un servidor con la ip
\texttt{78.47.140.228} se le puede asignar el nombre de dominio
\texttt{josetomastocino.com}, siendo ambos términos equivalentes.

El sistema \ac{DNS} se encarga de gestionar y distribuir las asociaciones entre
direcciones IP y nombres de dominio a través de los \textbf{registros DNS}. Para
cada dominio es posible definir varios tipos de registros DNS, con diferente
utilidad en cada caso. Los más habituales son los siguientes:

\begin{itemize}
\item Los registros de tipo \textbf{A} (\textit{address record}) se utilizan
  para traducir nombres de servidores a direcciones IPv4. Es el registro más habitual.
\item Los registros de tipo \textbf{AAAA} son similares a los tipo A pero para IPv6.
\item Los registros \textbf{CNAME} (\textit{canonical name}) se usan para crear
  nombres de servidor adicionales o \textit{alias}. Es la herramienta a utilizar
  cuando se desea añadir un subdominio, como \texttt{subdominio.uca.es}.
\item Los registros \textbf{MX} (\textit{mail exchange}) sirven para asociar un
  dominio a una lista de servidores de correo, de forma que un agente de correo
  sepa a qué servidor enviar los mensajes.
\item Los registros \textbf{TXT} (\textit{Text}) guardan información textual,
  utilizada habitualmente como forma de identificación frente a servicios
  externos. Por ejemplo, en el caso de Google Apps, los dueños de los dominios
  deben incluir un registro TXT con un contenido similar a

\begin{verbatim}
google-site-verification=rXOxyZounnZasA8Z7oaD3c14Jdj
\end{verbatim}
\end{itemize}

La información DNS asociada a un dominio se transmite a los clientes web a
través de una estructura jerárquica de servidores DNS. Cuando se produce un
cambio en los registros DNS de un dominio, estos cambios no están disponibles
inmediatamente a todos los usuarios, sino que se \textit{propagan} de forma
gradual a través todos los servidores de nombres del mundo. El proceso de
propagación puede durar entre 10 y 72 horas, dependiendo de la zona geográfica
de cada servidor.

El proceso de propagación de cambios, los errores en los servidores DNS y las
actividades maliciosas de terceros pueden dar lugar a que, en un momento dado,
un cliente no tenga la información DNS adecuada para un dominio. Puede ocurrir,
por ejemplo, que a través de un ataque de
\textit{man-in-the-middle}~\cite{ataquemitm} un usuario malicioso sirva a un
cliente unos registros DNS incorrectos para un dominio en particular,
redirigiendo a la víctima a otro servidor en el que puede producirse un robo de
datos. Es por tanto importante establecer un sistema de vigilancia que permita
comprobar que la información DNS asociada a un dominio sea la correcta en todo
momento, revisando periódicamente la información de los registros.

\subsection{Servicios similares}

En la actualidad existen bastantes servicios web que ofrecen algunas
características similares a las que se desean en SiteUp, pero no todas. Se
presentan a continuación algunos de estos servicios, junto a las características
que ofrecen y finalmente una comparativa con el proyecto que se presenta.

\paragraph{Pingdom} \url{http://pingdom.com}

Pingdom es un servicio sueco de monitorización fundado en 2007. Tienen más de
400.000 usuarios en 200 países, con clientes tan importantes como Spotify,
Instagram y Dropbox. Recientemente han actualizado su web con un nuevo diseño, y
han añadido soporte para iOS y Android. Entre sus servicios se encuentran los
clásicos chequeos por ping, por web, de protocolos y chequeos de interacciones
de varios pasos.

Ofrecen una cuenta gratuita con la que probar el servicio que permite añadir un
chequeo.

\paragraph{Alertra} \url{http://alertra.com}

Alertra es un servicio de monitorización fundado en el año 2000 en Estados
Unidos. Actualmente cuentan con bastantes tipos de chequeos, además de ofrecer
información técnica sobre los chequeos que se realizan.

La cuenta gratuita está limitada a 30 días, tras los cuales hay que pagar.


\paragraph{Uptime Robot} \url{http://uptimerobot.com}

Uptime Robot es una popular plataforma de monitorización desarrollada en Estados
Unidos por un solo programador. Su principal atractivo es su cuenta gratuita,
que permite a los usuarios añadir hasta 50 chequeos con una periodicidad de 5
minutos sin coste alguno. Lamentablemente en los últimos meses su servicio ha
ido teniendo problemas y su credibilidad ha caído, especialmente debido a no
contar con un equipo completo de personal técnico detrás del producto.

\paragraph{StatusCake} \url{http://statuscake.com}

StatusCake es uno de los principales servicios de monitorización de sitios web
actuales. Se ofrece (literalmente) como una alternativa mejor que Uptime Robot,
y cuenta con importantes clientes como Disney, Fender y Electronic Arts.

Su cuenta gratuita es una de las más generosas, ofreciendo chequeos ilimitados.

\subsubsection{Tabla comparativa}

En la figura~\ref{tabla_comparativa} se hace una comparativa de los requisitos
que se buscan en el proyecto, verificando si se ofrecen o no en las alternativas de
servicio previamente estudiadas.

\begin{sidewaystable}
  \centering
  \begin{tabular}{|l|c|c|c|c|c|}
    \hline
    \textit{Concepto} & Pingdom & Alertra & UptimeRobot & StatusCake & SiteUp\\
    \hline
    Chequeo mediante Ping & Sí & Sí & Sí & Sí & Sí \\
    \hline    
    Chequeo de peticiones HTTP (código de estado) & No & No & No & Sí & Sí \\
    \hline
    Chequeo de peticiones HTTP (contenido) & Sí & Sí & Sí & Solo cuentas de pago & Sí \\
    \hline
    Chequeo de registros DNS & Limitado & No & No & Limitado & Sí \\
    \hline
    Chequeo de puertos & Sí & Sí & Sí & Sí & Sí \\
    \hline
    Chequeo de puertos (contenido) & Sí & Sí & No & No & Sí \\
    \hline
    Cuenta gratuita & Sí & 30 días & Sí & Sí & Sí \\
    \hline    
    Chequeos en cuenta gratuita & 1 & 1 & 50 & Ilimitado & Ilimitado \\
    \hline
    Frecuencia mínima en cuenta gratuita & 1 minuto & 5 minutos & 5 minutos & 5 minutos & 1 minuto \\
    \hline
    Notificación por e-mail & Sí & Sí & Sí & Sí & Sí \\
    \hline
    Notificación por App Android & Sí & No & No & No & Sí \\
    \hline
    Open Source & No & No & No & No & Sí \\
    \hline   
  \end{tabular}
  \caption{Tabla comparativa de servicios de monitorización}
  \label{tabla_comparativa}
\end{sidewaystable}

Una carencia importante bastante generalizada es la falta de aplicaciones
nativas para móviles con las que recibir notificaciones sobre el estado de los
servicios y gestionar los chequeos. Solo Pingdom y el presente proyecto, SiteUp,
ofrecen aplicaciones para Android.

Así pues, queda patente la dificultad de encontrar un servicio que aúne el mayor
número de características posibles a la vez que mantiene un servicio gratuito y
de calidad, quedando manifiesta la necesidad del proyecto.


\section{Alcance}
\textbf{SiteUp} se modela como una herramienta de monitorización de servicios de
Internet accesible a través de la web. Los usuarios tendrán la posibilidad de
crear y gestionar una serie de \textit{chequeos} de diversos tipos sobre los
servicios web que elijan. La aplicación irá recopilando información relativa a
esos chequeos, e informará al usuario en caso de que las verificaciones que se
hayan dado de alta no coincidan con los resultados esperados.

Además, el usuario tendrá la posibilidad de recibir notificaciones de manera
instantánea a través del correo electrónico y de una aplicación para la
plataforma móvil \textbf{Android}. El servicio web estará totalmente adaptado
para su uso en dispositivos móviles.



\subsection{Licencia}
El proyecto está publicado como software libre bajo la licencia
\ac{GPL} versión 3. El conjunto de bibliotecas y módulos utilizados
tienen las siguientes licencias:

\begin{itemize}
\item El intérprete del lenguaje de programación \textbf{Python} se distribuye
  bajo la licencia \ac{PSFL}, una licencia permisiva estilo \ac{BSD} y
  compatible con la \ac{GNU} \ac{GPL}.

\item \textbf{Django}~\cite{django}, el framework de desarrollo web en el que se
  basa la aplicación, se distribuye bajo la licencia \textit{\ac{BSD}}.

\item El servidor web \textbf{nginx}~\cite{nginx} está licenciado bajo la
  licencia \ac{BSD} simplificada.

\item Los siguientes paquetes de Python utilizan también la licencia \ac{BSD}:
  \begin{itemize}
  \item Celery
  \item Sqlparse
  \item iPython
  \item dnspython
  \item coverage
  \item django-rest-framework
  \item billiard
  \item anyjson
  \item Fabric
  \end{itemize}

\item Los siguientes paquetes de Python utilizan la licencia \ac{MIT}:
  \begin{itemize}
  \item PyDot
  \item Gunicorn
  \item Requests
  \item django-extensions
  \item six
  \item sh
  \item pip
  \item virtualenvwrapper
  \item factory-boy
  \end{itemize}

\end{itemize}

\section{Estructura del documento}
El presente documento se rige según la siguiente estructura:

\begin{itemize}
\item \textbf{\nameref{chap:introduccion}}. Se exponen el contexto, las motivaciones y
  objetivos detrás del proyecto \textbf{SiteUp}, así como información sobre las
  licencias de sus componentes, glosario y estructura del documento.

\item \textbf{\nameref{chap:calendario}}, donde se explica la planificación del
  proyecto, la división de sus etapas, la extensión de las etapas a lo largo del
  tiempo y los porcentajes de esfuerzo.

\item \textbf{\nameref{chap:requisitos}}, capítulo en el que se formalizan los
  objetivos y requisitos planteados en la introducción.

\item \textbf{\nameref{chap:analisis}}. Se detalla la fase de análisis del
  sistema, explicando los requisitos funcionales del sistema, los diferentes
  casos de uso y bosquejando las interfaces visuales de los diferentes sistemas.

\item \textbf{\nameref{chap:diseno}}. Tras el análisis, se expone en detalle
  la etapa de diseño del sistema, con los diagramas de las arquitecturas lógicas
  de los sistemas y los diagramas de diseño físico de datos.

\item \textbf{\nameref{chap:implementacion}}. Una vez analizado el sistema y
  definido su diseño, en esta parte se detallan las decisiones de implementación
  más relevantes que tuvieron lugar durante el desarrollo del proyecto.

\item \textbf{\nameref{chap:pruebas}}. Se describen las pruebas que se han
  llevado a cabo sobre el proyecto para garantizar su fiabilidad y consistencia.

\item \textbf{\nameref{chap:conclusiones}}. Comentamos las conclusiones a las
  que se han llegado durante el transcurso y al término del proyecto.
\end{itemize}

Y los siguientes apéndices:
\begin{itemize}
\item \textbf{\nameref{chap:manual_instalacion}} en sistemas nuevos.
\item \textbf{\nameref{chap:manual_usuario}}, donde se explica cómo usar la aplicación.
\end{itemize}

\section{Acrónimos}

\begin{acronym}
  \acro{IANA}{Internet Assigned Numbers Authority}
  \acro{IETF}{Internet Engineering Task Force}
  \acro{Sass}{Syntactically Awesome Style Sheets}
  \acro{AJAX}{Asynchronous JavaScript and XML}
  \acro{SQL}{Structured Query Language}
  \acro{ORM}{Object-relational mapping}
  \acro{AMQP}{Advanced Message Queuing Protocol}
  \acro{LOPD}{Ley Orgánica de Protección de Datos}
  \acro{WSGI}{Web Server Gateway Interface}
  \acro{PSFL}{Python Software Foundation License}
  \acro{ABI}{Application Binary Interface}
  \acro{ACL2s}{ACL2 Sedan}
  \acro{ACL2}{A Computational Logic for Applicative Common Lisp}
  \acro{ALSA}{Advanced Linux Sound Architecture}
  \acro{AMD}{Advanced Micro Devices}
  \acro{API}{Application Programming Interface}
  \acro{APL}{Apache Public License}
  \acro{ASCII}{American Standard Code for Information Interchange}
  \acro{BNF}{Backus-Naur Form}
  \acro{BOM}{Byte Order Mark}
  \acro{BSD}{Berkeley Software Distribution}
  \acro{C3}{Chrysler Comprehensive Compensation System}
  \acro{CAP}{Complex Arithmetic Processor}
  \acro{CASE}{Computer Assisted Software Engineering}
  \acro{CRUD}{Create-Read-Update-Delete}
  \acro{CD}{Compact Disc}
  \acro{CPAN}{Comprehensive Perl Archive Network}
  \acro{CPU}{Central Processing Unit}
  \acro{CSS}{Cascading Style Sheet(s)}
  \acro{CVS}{Concurrent Versions System}
  \acro{DDR}{Dance Dance Revolution}
  \acro{DNS}{Domain Name Server}
  \acro{DOM}{Document Object Model}
  \acro{DFT}{Discrete Fourier Transform}
  \acro{DSP}{Digital Signal Processing}
  \acro{DTD}{Do\-cu\-ment Type De\-fi\-ni\-tion}
  \acro{EAFP}{Easier to Ask for Forgiveness than Permission}
  \acro{FDL}{Free Documentation License}
  \acro{FFT}{Fast Fourier Transform}
  \acro{FSF}{Free Software Foundation}
  \acro{GCC}{GNU Compiler Collection}
  \acro{GCJ}{GNU Compiler for Java}
  \acro{GCL}{GNU Common Lisp}
  \acro{GIMP}{\acs{GNU} Image Manipulation Program}
  \acro{GML}{Generalized Markup Language}
  \acro{GNU}{GNU is Not Unix}
  \acro{GPL}{General Public License}
  \acro{GTK+}{\acs{GIMP} Toolkit}
  \acro{GUI}{Graphical User Interface}
  \acro{HTML}{Hyper Text Markup Language}
  \acro{HTTP}{Hyper Text Transfer Protocol}
  \acro{IBM}{International Business Machines}
  \acro{ICMP}{Internet Control Message Protocol}
  \acro{IDE}{Integrated Development Environment}
  \acro{IEEE}{Institute of Electrical and Electronics Engineers}
  \acro{IIS}{Internet Information Server}
  \acro{ISP}{Internet Service Provider}
  \acro{IP}{Internet Protocol}
  \acro{J2SE}{Java 2 Standard Edition}
  \acro{JACK}{JACK Audio Connection Kit}
  \acro{JAXP}{Java API for XML Parsing}
  \acro{JDBC}{Java DataBase Connectivity}
  \acro{JDK}{Java Development Kit}
  \acro{JIT}{Just In Time}
  \acro{JRE}{Java Runtime Environment}
  \acro{JSON}{JavaScript Object Notation}
  \acro{JSR}{Java Specification Request}
  \acro{JVM}{Java Virtual Machine}
  \acro{LGPL}{Lesser General Public License}
  \acro{LTS}{Long Time Support}
  \acro{MDI}{Multiple Document Interface}
  \acro{MIDI}{Musical Instrument Digital Interface}
  \acro{MIT}{Massachusetts Institute of Technology}
  \acro{MP3}{MPEG-2 Audio Layer III}
  \acro{MVC}{Model View Controller}
  \acro{MVP}{Model View Presenter}
  \acro{ODF}{Open Document Format}
  \acro{OSS}{Open Sound System}
  \acro{PAR}{Perl ARchive Toolkit}
  \acro{PCM}{Pulse-Code Modulation}
  \acro{PDF}{Portable Document Format}
  \acro{POD}{Plain Old Do\-cu\-men\-ta\-tion}
  \acro{POSIX}{Portable Operating System Interface, UniX based}
  \acro{Perl}{Practical Extraction and Report Language}
  \acro{RPM}{RPM Package Manager}
  \acro{RUP}{Rational Unified Process}
  \acro{RTP}{Real-time Transport Protocol}
  \acro{SAX}{Simple API for XML}
  \acro{SDI}{Single Document Interface}
  \acro{SDL}{Simple DirectMedia Layer}
  \acro{SGML}{Standard Generalized Markup Language}
  \acro{SSL}{Secure Sockets Layer}
  \acro{SVG}{Structured Vector Graphics}
  \acro{SWT}{Standard Widget Toolkit}
  \acro{TCK}{Technology Compatibility Kit}
  \acro{TDI}{Tabbed Document Interface}
  \acro{UML}{Unified Modelling Language}
  \acro{URI}{Uniform Resource Identifier}
  \acro{URL}{Uniform Resource Locator}
  \acro{UTF}{Universal Transformation Format}
  \acro{VPS}{Virtual Private Server}
  \acro{W3C}{World Wide Web Consortium}
  \acro{WWW}{World Wide Web}
  \acro{XHTML}{eXtensible Hyper Text Markup Language}
  \acro{XML}{eXtensible Markup Language}
  \acro{XP}{eXtreme Programming}
  \acro{XSL-FO}{eXtensible Stylesheet Language Formatting Objects}
  \acro{XSLT}{eXtensible Stylesheet Language Transformations}
  \acro{XSL}{eXtensible Stylesheet Language}
  \acro{YAML}{YAML Ain't a Markup Language}
  \acro{YAXML}{YAML XML binding}
\end{acronym}



%%% Local Variables:
%%% mode: latex
%%% TeX-master: "../memoria"
%%% End:
