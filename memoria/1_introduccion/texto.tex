\section{Contexto social}

Las tecnologías de la información en general e Internet en particular son ya
parte integral de la sociedad. Casi todos los ámbitos de la vida,
desde las interacciones sociales hasta la búsqueda de empleo, cuentan ya con
su reflejo en las tecnologías de la información, a menudo mediante el uso de
servicios web a través de Internet.

No solo los aspectos tradicionales de la sociedad tienen su presencia en las
redes, también han surgido nuevos modelos empresariales propios de Internet que
han crecido de manera importante y se han situado a niveles comparables a los de
las empresas tradicionales. Empresas puramente digitales como Facebook o Twitter
ya cotizan en bolsa y realizan operaciones bursátiles del orden de miles de
millones de dólares~\cite{facebook-acquires-whatsapp}.

Se pone pues de manifiesto la importancia de la fiablidad de los servicios e
infraestructuras de los que dependen estos nuevos modelos de negocio. El
\textit{uptime} -- un término inglés que describe el porcentaje de tiempo que un
servicio se mantiene disponible -- debe ser siempre cercano al 100\%, dado que
en caso contrario los potenciales usuarios del servicio se encontrarán con que
no pueden acceder a él, dando lugar incluso a pérdidas económicas. Es el caso de
Amazon, que llegó a perder 4.8 millones de dólares al sufrir un fallo que dejó
inaccesible su web durante 40 minutos~\cite{amazon}.

De todo lo expuesto se extrae la necesidad de contar con sistemas para
monitorizar la disponibilidad de estos servicios y, en caso necesario, actuar de
manera que puedan subsanarse las causas de los problemas.

\section{Motivación}
\label{sec:situacion-actual}

La idea de desarrollar este proyecto surge a raíz de una necesidad
personal del desarrollador del proyecto. 

En octubre de 2013 tomé parte en un un importante proceso de selección en el que
las comunicaciones se estaban llevando a cabo a través de correo electrónico. En
particular, la cuenta de correo electrónico que estaba usando
(\texttt{info@josetomastocino.com}) tenía un dominio personalizado y estaba
gestionada a través de Google Apps~\cite{google-apps}. Esto era posible gracias
a que en el dominio se dieron de alta unos registros de tipo MX y TXT que hacían
que el correo se redirigiese a los servidores de Google.

A mitad del proceso de selección, la empresa que gestionaba el dominio de mi web
personal (y, por extensión, mi correo electrónico) tuvo un problema y reemplazó
los registros DNS personalizados por unos por defecto. En particular, los
registros MX y TXT quedaron en blanco, y los registros A apuntaron a una página
de párking\footnote{Una página de \textit{párking de dominio} sirve como página
  temporal mientras un servicio web no es lanzado, de forma que el dominio no
  esté en blanco.}, de forma que los chequeos de tipo ping que tenía puestos no
dieron error, ya que efectivamente el servidor devolvía el ping, pero el
contenido de los registros no era el correcto, ni los correos se estaban
direccionando bien.

Esta situación se prolongó durante varios días, en los cuales perdí varios
mensajes importantes. Esto podría haberse detectado rápidamente si hubiera tenido
alguna clase de vigilancia que verificase que los registros DNS tenían el
contenido correcto. 

De ahí nace la necesidad de realizar un estudio de las alternativas existentes
para este tipo de vigilancia y, al concluir que no existía ninguna alternativa
libre, se decide iniciar el proyecto SiteUp para la vigilancia de servicios de
internet.


\section{Objetivos}
\label{sec:objetivos}

A la hora de definir los objetivos de un sistema, podemos agruparlos
en dos tipos diferentes: \textbf{funcionales} y
\textbf{transversales}. Los primeros se refieren a \textit{qué} debe
hacer la aplicación que vamos a desarrollar, e inciden
directamente en la experiencia del usuario y de potenciales
desarrolladores.

Por otro lado, los objetivos transversales son aquellos invisibles al
usuario final, pero que de forma inherente actúan sobre el resultado
final de la aplicación y sobre la experiencia de desarrollo de la misma.

\subsection{Funcionales}
\begin{itemize}
\item Crear un conjunto de herramientas para la monitorización y el chequeo de
  diversos aspectos del estado de un servicio de Internet.
\item Crear una aplicación online, de acceso público, que permita la creación y
  gestión de chequeos de manera sencilla, basada internamente en las
  herramientas mencionadas en el punto anterior.
\item Habilitar a esta aplicación de un sistema de notificaciones mediante correo
  electrónico que alerte a los usuarios de posibles cambios en la disponibilidad
  de los servicios monitorizados.
\item Crear una aplicación móvil para que los usuarios tengan la opción de
  recibir notificaciones instantáneas provenientes de la aplicación web con
  información de sus chequeos.
\end{itemize}

\subsection{Transversales}
\begin{itemize}
\item Investigar y conocer los vectores de vigilancia usados habitualmente para
  monitorizar servicios de Internet.
\item Ampliar mis conocimientos sobre desarrollo web en general y las
  tecnologías de back-end en particular.
\item Adquirir soltura en el uso del lenguaje de programación Python en entornos
  web.
\item Obtener una base de conocimientos mínima sobre el desarrollo de
  aplicaciones sobre la plataforma móvil Android.
\item Utilizar un enfoque de análisis, diseño y codificación orientado
  a objetos, de una forma lo más clara y modular posible, para
  permitir ampliaciones y modificaciones sobre la aplicación por
  terceras personas.
\item Hacer uso de herramientas básicas en el desarrollo de software,
  como son los \textbf{Sistemas de Control de Versiones} para llevar
  un control realista del desarrollo del software, así como hacer de
  las veces de sistema de copias de seguridad.
\end{itemize}

\section{Estado del arte}
\label{sec:estado-del-arte}

En la actualidad existen bastantes servicios web que ofrecen algunas
características similares a las que se desean en SiteUp, pero no todas. Se
presentan a continuación algunos de estos servicios, junto a los problemas que
se han detectado.

\begin{description}
\item[Pingdom] \url{http://pingdom.com} -- La opción más veterana y popular, con
  clientes muy importantes. La cuenta gratuita solo permite añadir un check. No
  ofrece chequeo de registros DNS.
\item[Alertra] \url{http://alertra.com} -- La cuenta gratuita solo dura 30
  días. No ofrece chequeo de registros DNS.
\item[UptimeRobot.com] \url{http://uptimerobot.com} -- Periodicidad mínima de 5
  minutos. No ofrece chequeo de DNS.
\item[ServerCheck] \url{http://servercheck.in} -- Aspecto amateur. No tienen
  cuenta gratuita. No tienen chequeo de DNS.
\item[StatusCake] \url{http://statuscake.com} -- Periodicidad mínima de 5
  minutos. No permite el chequeo de códigos de estado en su cuenta gratuita.

\end{description}

Una carencia importante que no se ha mencionado por ser generalizada es la falta
de aplicaciones nativas para móviles con las que recibir notificaciones sobre el
estado de los servicios y gestionar los chequeos.

Así pues, queda patente la dificultad de encontrar un servicio que aúne el mayor
número de características posibles a la vez que mantiene un servicio gratuito y
de calidad, quedando manifiesta la necesidad del proyecto.


\section{Alcance}
\textbf{SiteUp} se modela como una herramienta de monitorización de servicios de
internet accesible a través de la web. Los usuarios tendrán la posibilidad de
crear y gestionar una serie de \textit{chequeos} de diversos tipos sobre los
servicios web que elijan. La aplicación irá recopilando información relativa a
esos chequeos, e informará al usuario en caso de que las verificaciones que se
hayan dado de alta no coincidan con los resultados esperados.

Además, el usuario tendrá la posibilidad de recibir notificaciones de manera
instantánea a través del correo electrónico y de una aplicación para la
plataforma móvil \textbf{Android}. El servicio web estará totalmente adaptado
para su uso en dispositivos móviles.

\subsection{Limitaciones del proyecto}
Aunque cubre una gran parte de los puntos de vigilancia habituales, la
aplicación se limita a ofrecer chequeo de respuesta de ping, chequeo de puertos,
chequeo de registros DNS y chequeo de cabeceras y contenidos HTTP.

La aplicación de Android no contará con ninguna funcionalidad para gestionar los
chequeos de un usuario, sino que servirá para recibir notificaciones instantáneas
provenientes de la aplicación web. Ésta, por otro lado, estará completamente
adaptada para su uso a través de dispositivos móviles gracias al uso del
\textit{responsive web design}.

Idealmente los chequeos deberían hacerse simultáneamente desde diferentes
máquinas colocadas en diversos puntos geográficos, para así tener unos
resultados más fiables. La falta de infraestructuras y la finalidad didáctica
del proyecto limitan la aplicación a una estructura monolítica en la que los
chequeos se hacen desde una sola máquina, la misma que sirve el servicio web.

\subsection{Licencia}
El proyecto está publicado como software libre bajo la licencia
\ac{GPL} versión 3. El conjunto de bibliotecas y módulos utilizados
tienen las siguientes licencias:

\begin{itemize}
\item El intérprete del lenguaje de programación \textbf{Python} se distribuye
  bajo la licencia \ac{PSFL}, una licencia permisiva estilo \ac{BSD} y
  compatible con la \ac{GNU} \ac{GPL}.

\item \textbf{Django}~\cite{django}, el framework de desarrollo web en el que se
  basa la aplicación, se distribuye bajo la licencia \textit{\ac{BSD}}.

\item El servidor web \textbf{nginx}~\cite{nginx} está licenciado bajo la
  licencia \ac{BSD} simplificada.

\item Los siguientes paquetes de Python utilizan también la licencia \ac{BSD}:
  \begin{itemize}
  \item Celery
  \item Sqlparse
  \item iPython
  \item dnspython
  \item coverage
  \item django-rest-framework
  \item billiard
  \item anyjson
  \item Fabric
  \end{itemize}

\item Los siguientes paquetes de Python utilizan la licencia \ac{MIT}:
  \begin{itemize}
  \item PyDot
  \item Gunicorn
  \item Requests
  \item django-extensions
  \item six
  \item sh
  \item pip
  \item virtualenvwrapper
  \item factory-boy
  \end{itemize}

\end{itemize}

\section{Estructura del documento}
El presente documento se rige según la siguiente estructura:

\begin{itemize}
\item \textbf{\nameref{chap:introduccion}}. Se exponen el contexto, las motivaciones y
  objetivos detrás del proyecto \textbf{SiteUp}, así como información sobre las
  licencias de sus componentes, glosario y estructura del documento.

\item \textbf{\nameref{chap:calendario}}, donde se explica la planificación del
  proyecto, la división de sus etapas, la extensión de las etapas a lo largo del
  tiempo y los porcentajes de esfuerzo.

\item \textbf{\nameref{chap:requisitos}}, capítulo en el que se formalizan los
  objetivos y requisitos planteados en la introducción.

\item \textbf{\nameref{chap:analisis}}. Se detalla la fase de análisis del
  sistema, explicando los requisitos funcionales del sistema, los diferentes
  casos de uso y bosquejando las interfaces visuales de los diferentes sistemas.

\item \textbf{\nameref{chap:diseno}}. Tras el análisis, se expone en detalle
  la etapa de diseño del sistema, con los diagramas de las arquitecturas lógicas
  de los sistemas y los diagramas de diseño físico de datos.

\item \textbf{\nameref{chap:implementacion}}. Una vez analizado el sistema y
  definido su diseño, en esta parte se detallan las decisiones de implementación
  más relevantes que tuvieron lugar durante el desarrollo del proyecto.

\item \textbf{\nameref{chap:pruebas}}. Listamos y describimos las pruebas que se
  han llevado a cabo sobre el proyecto para garantizar su fiabilidad y
  consistencia.

\item \textbf{\nameref{chap:conclusiones}}. Comentamos las conclusiones a las
  que se han llegado durante el transcurso y al término del proyecto.
\end{itemize}

Y los siguientes apéndices:
\begin{itemize}
\item \textbf{\nameref{chap:manual_instalacion}} en sistemas nuevos.
\item \textbf{\nameref{chap:manual_usuario}}, donde se explica cómo usar la aplicación.
\end{itemize}

\section{Acrónimos}

\begin{acronym}
  \acro{IANA}{Internet Assigned Numbers Authority}
  \acro{IETF}{Internet Engineering Task Force}
  \acro{Sass}{Syntactically Awesome Style Sheets}
  \acro{AJAX}{Asynchronous JavaScript and XML}
  \acro{SQL}{Structured Query Language}
  \acro{ORM}{Object-relational mapping}
  \acro{AMQP}{Advanced Message Queuing Protocol}
  \acro{LOPD}{Ley Orgánica de Protección de Datos}
  \acro{WSGI}{Web Server Gateway Interface}
  \acro{PSFL}{Python Software Foundation License}
  \acro{ABI}{Application Binary Interface}
  \acro{ACL2s}{ACL2 Sedan}
  \acro{ACL2}{A Computational Logic for Applicative Common Lisp}
  \acro{ALSA}{Advanced Linux Sound Architecture}
  \acro{AMD}{Advanced Micro Devices}
  \acro{API}{Application Programming Interface}
  \acro{APL}{Apache Public License}
  \acro{ASCII}{American Standard Code for Information Interchange}
  \acro{BNF}{Backus-Naur Form}
  \acro{BOM}{Byte Order Mark}
  \acro{BSD}{Berkeley Software Distribution}
  \acro{C3}{Chrysler Comprehensive Compensation System}
  \acro{CAP}{Complex Arithmetic Processor}
  \acro{CASE}{Computer Assisted Software Engineering}
  \acro{CRUD}{Create-Read-Update-Delete}
  \acro{CD}{Compact Disc}
  \acro{CPAN}{Comprehensive Perl Archive Network}
  \acro{CPU}{Central Processing Unit}
  \acro{CSS}{Cascading Style Sheet(s)}
  \acro{CVS}{Concurrent Versions System}
  \acro{DDR}{Dance Dance Revolution}
  \acro{DNS}{Domain Name Server}
  \acro{DOM}{Document Object Model}
  \acro{DFT}{Discrete Fourier Transform}
  \acro{DSP}{Digital Signal Processing}
  \acro{DTD}{Do\-cu\-ment Type De\-fi\-ni\-tion}
  \acro{EAFP}{Easier to Ask for Forgiveness than Permission}
  \acro{FDL}{Free Documentation License}
  \acro{FFT}{Fast Fourier Transform}
  \acro{FSF}{Free Software Foundation}
  \acro{GCC}{GNU Compiler Collection}
  \acro{GCJ}{GNU Compiler for Java}
  \acro{GCL}{GNU Common Lisp}
  \acro{GIMP}{\acs{GNU} Image Manipulation Program}
  \acro{GML}{Generalized Markup Language}
  \acro{GNU}{GNU is Not Unix}
  \acro{GPL}{General Public License}
  \acro{GTK+}{\acs{GIMP} Toolkit}
  \acro{GUI}{Graphical User Interface}
  \acro{HTML}{Hyper Text Markup Language}
  \acro{HTTP}{Hyper Text Transfer Protocol}
  \acro{IBM}{International Business Machines}
  \acro{ICMP}{Internet Control Message Protocol}
  \acro{IDE}{Integrated Development Environment}
  \acro{IEEE}{Institute of Electrical and Electronics Engineers}
  \acro{IIS}{Internet Information Server}
  \acro{ISP}{Internet Service Provider}
  \acro{IP}{Internet Protocol}
  \acro{J2SE}{Java 2 Standard Edition}
  \acro{JACK}{JACK Audio Connection Kit}
  \acro{JAXP}{Java API for XML Parsing}
  \acro{JDBC}{Java DataBase Connectivity}
  \acro{JDK}{Java Development Kit}
  \acro{JIT}{Just In Time}
  \acro{JRE}{Java Runtime Environment}
  \acro{JSON}{JavaScript Object Notation}
  \acro{JSR}{Java Specification Request}
  \acro{JVM}{Java Virtual Machine}
  \acro{LGPL}{Lesser General Public License}
  \acro{LTS}{Long Time Support}
  \acro{MDI}{Multiple Document Interface}
  \acro{MIDI}{Musical Instrument Digital Interface}
  \acro{MIT}{Massachusetts Institute of Technology}
  \acro{MP3}{MPEG-2 Audio Layer III}
  \acro{MVC}{Model View Controller}
  \acro{MVP}{Model View Presenter}
  \acro{ODF}{Open Document Format}
  \acro{OSS}{Open Sound System}
  \acro{PAR}{Perl ARchive Toolkit}
  \acro{PCM}{Pulse-Code Modulation}
  \acro{PDF}{Portable Document Format}
  \acro{POD}{Plain Old Do\-cu\-men\-ta\-tion}
  \acro{POSIX}{Portable Operating System Interface, UniX based}
  \acro{Perl}{Practical Extraction and Report Language}
  \acro{RPM}{RPM Package Manager}
  \acro{RUP}{Rational Unified Process}
  \acro{RTP}{Real-time Transport Protocol}
  \acro{SAX}{Simple API for XML}
  \acro{SDI}{Single Document Interface}
  \acro{SDL}{Simple DirectMedia Layer}
  \acro{SGML}{Standard Generalized Markup Language}
  \acro{SSL}{Secure Sockets Layer}
  \acro{SVG}{Structured Vector Graphics}
  \acro{SWT}{Standard Widget Toolkit}
  \acro{TCK}{Technology Compatibility Kit}
  \acro{TDI}{Tabbed Document Interface}
  \acro{UML}{Unified Modelling Language}
  \acro{URI}{Uniform Resource Identifier}
  \acro{URL}{Uniform Resource Locator}
  \acro{UTF}{Universal Transformation Format}
  \acro{VPS}{Virtual Private Server}
  \acro{W3C}{World Wide Web Consortium}
  \acro{WWW}{World Wide Web}
  \acro{XHTML}{eXtensible Hyper Text Markup Language}
  \acro{XML}{eXtensible Markup Language}
  \acro{XP}{eXtreme Programming}
  \acro{XSL-FO}{eXtensible Stylesheet Language Formatting Objects}
  \acro{XSLT}{eXtensible Stylesheet Language Transformations}
  \acro{XSL}{eXtensible Stylesheet Language}
  \acro{YAML}{YAML Ain't a Markup Language}
  \acro{YAXML}{YAML XML binding}
\end{acronym}



%%% Local Variables:
%%% mode: latex
%%% TeX-master: "../memoria"
%%% End:
