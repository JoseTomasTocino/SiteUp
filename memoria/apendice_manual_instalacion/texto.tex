A continuación se presentan las instrucciones de instalación de la plataforma
web en un servidor de producción.

Las instrucciones de instalación que se presentan a continuación suponen un
sistema con unos requisitos de hardware mínimos especificados en la
sección~\ref{subsec:entorno-produccion},
\textit{\nameref{subsec:entorno-produccion}}. Se supone un servidor con una
distribución basada en paquetería Debian con acceso root.

La versión más actualizada de las instrucciones de instalación de la plataforma
web se encuentran en el fichero \texttt{INSTALLING-web.md} alojado en el
directorio raíz de la forja de código~\cite{forja}.

\section{Instalación inicial}

\subsection{Descarga de código}

Para instalar la plataforma web es necesario clonar el repositorio~\cite{forja}
con el código de la aplicación. Primero, creamos una ubicación donde alojar el
código desde la terminal:

\begin{bashcode}
$ mkdir /srv/siteup -P  
$ cd /srv/siteup
\end{bashcode}

Una vez ahí, clonamos el repositorio desde Github:

\begin{bashcode}
$ git clone https://github.com/JoseTomasTocino/pfc-ii.git .  
\end{bashcode}

\subsection{Entorno virtual y dependencias}

Como se ha comentado previamente, el proyecto utiliza \textbf{virtualenv} y
\textbf{virtualenvwrapper} para una gestión más limpia de las dependencias. La
instalación de estos dos elementos es sencilla, en primer lugar se instalan los
dos paquetes:

\begin{bashcode}
$ pip install virtualenv
$ pip install virtualenvwrapper
\end{bashcode}

En segundo lugar, añadimos el código de \textit{bootstrapping} de
virtualenvwrapper a nuestro perfil de terminal, habitualmente \texttt{.bashrc}:

\begin{bashcode}
$ cat >> ~/.bashrc

if [ -f /usr/local/bin/virtualenvwrapper.sh ]
then
    source /usr/local/bin/virtualenvwrapper.sh
fi

EOF
\end{bashcode}

Hecho esto, será necesario reiniciar la terminal, tras lo cual podremos crear el
entorno virtual e instalar las dependencias:

\begin{bashcode}
$ mkvirtualenv siteup
$ sudo apt-get install python-dev
$ pip install -r web/requirements.txt 
\end{bashcode}

\subsection{Creación de la base de datos}

El siguiente paso es preparar la base de adtos. El siguiente comando genera la
estructura básica de la base de datos para las tablas y aplica las migraciones
existentes:

\begin{bashcode}
$ python web/manage.py syncdb
$ python web/migrate siteup_api  
\end{bashcode}

Tras esto, la base de datos se habrá generado y se encontrará en el fichero\\
\texttt{web/db/django-db.sqlite3}.

\subsection{Instalación del servidor}

En estas instrucciones vamos a usar \textbf{Nginx} como servidor frontal, aunque
también es posible utilizar Apache. Para instalar nginx, en sistemas GNU/Linux
basados en paquetería Debian, el procedimiento es muy sencillo:

\begin{bashcode}
$ sudo apt-get install nginx -y
\end{bashcode}

\subsection{Instalación del bróker de mensajes}

El bróker de mensajes elegido es \textbf{RabbitMQ}, aunque en principio valdría
cualquier otro bróker que fuese compatible con el estándar \ac{AMQP}. La
instalación es sencilla:

\begin{bashcode}
$ sudo apt-get install rabbitmq-server -y
\end{bashcode}

La configuración por defecto es suficiente para las necesidades del proyecto.

\subsection{Ejecución de los servicios}

Para el control de los servicios que ejecutan el proyecto es necesario utilizar
alguna clase de software de monitorización de procesos. La opción recomendada es
\textbf{supervisor}. Su instalación es sencilla:

\begin{bashcode}
$ sudo apt-get install supervisor -y
\end{bashcode}

Tras esto, es necesario generar el fichero de configuración de
supervisor. SiteUp integra una tarea que automáticamente genera este fichero por
nosotros. Solo tendremos que indicar un par de variables, que nos pedirá por
teclado. En particular:

\begin{itemize}
\item \texttt{GMAIL\_PASS}, es la clave de la cuenta de correo utilizada para
  enviar notificaciones. Se define en el fichero \texttt{web/siteup/settings/base.py}.
\item \texttt{GCM\_API\_KEY}, es la clave de la API del servicio Google Cloud Messaging.
\end{itemize}

Para generar el fichero de configuración utilizaremos el siguiente comando:

\begin{bashcode}
$ python web/manage.py supervisor_conf
\end{bashcode}

Se generará un fichero \texttt{supervisor-siteup.conf} con la configuración. Es
buena idea revisar el fichero para comprobar que todas las rutas son
correctas. Tras ello, solo queda mover el fichero al lugar correcto y reiniciar
el demonio de supervisor.

\begin{bashcode}
$ sudo mv web/supervisor-siteup.conf /etc/supervisord/conf.d  
\end{bashcode}

En entornos de producción, normalmente no se usará supervisor, ni gunicorn ni
nginx, sino que utilizaremos el servidor integrado de Django. En esas
circunstancias será necesario definir las variables mencionadas previamente de
otra manera. La manera más habitual es añadirlas al \textbf{script de activación} del
entorno virtual, de la siguiente manera:

\begin{bashcode}
$ cdvirtualenv
$ cd bin
$ cat >> postactivate
export GMAIL_PASS=yourpass
export GCM_API_KEY=yourkey

EOF
\end{bashcode}

\subsection{Configuración del servidor frontal}

El siguiente paso es configurar el servidor nginx para que reciba las peticiones
apropiadas. Vamos a suponer que SiteUp va a recibir las peticiones en el dominio
\texttt{siteup.uca.es}. Creamos el fichero
\texttt{/etc/nginx/sites-available/siteup} y añadimos el siguiente contenido:

\begin{minted}{nginx}
 server {
    listen 80;
    server_name siteup.uca.es;

    # Static files
    location /static/ {
        root /srv/siteup/web/siteup_frontend;
        try_files $uri $uri/ =404;
    }

    location / {
        proxy_set_header X-Real-IP        $remote_addr;
        proxy_set_header X-Forwarded-For  $proxy_add_x_forwarded_for;
        proxy_set_header Host             $http_host;
        proxy_redirect off;

        if (!-f $request_filename) {
            proxy_pass       http://127.0.0.1:8000;
            break;
        }
    }
} 
\end{minted}

Tras eso, activamos el fichero de configuración de la siguiente manera:

\begin{bashcode}
$ cd /etc/nginx/sites-enabled
$ ln -s ../sites-available/siteup
\end{bashcode}

Básicamente lo que hace ese fichero de configuración es revisar las peticiones
que llegan y, si apuntan a la ruta \texttt{/static}, nginx se encarga de servir
esos ficheros estáticos de front-end. Si no, pasa la petición al servidor
gunicorn que estará escuchando en la dirección local \texttt{127.0.0.1:8000}.

\subsection{Lanzamiento final}

Tras toda esta configuración, los últimos pasos que quedan son activar los
servicios y servidores. Primero, lanzamos el controlador supervisor, que a su
vez lanzará el servidor dinámico gunicorn y el servidor de tareas celery:

\begin{bashcode}
$ sudo service supervisord reload  
\end{bashcode}

Seguidamente, reiniciamos el servidor frontal para que pueda empezar a recibir
peticiones:

\begin{bashcode}
$ sudo service nginx restart  
\end{bashcode}

Con esto, el proyecto estará funcionando y debería ser accesible a través del
navegador.


\section{Mejora del rendimiento}

La configuración por defecto de \textbf{RabbitMQ} hace que consuma mucha memoria
y puede llegar a resultar un problema en sistemas con recursos limitados, como
los VPS de bajo coste.

Para arreglar esto, una opción es modificar el fichero de configuración,
situadio en \texttt{/etc/rabbitmq/rabbitmq.config} para limitar la memoria
máxima usada por Rabbit a un 15\% del total. El contenido a añadir al fichero es
el siguiente:

\begin{minted}{text}
[
    {rabbit, [{vm_memory_high_watermark, 0.15}]}
].  
\end{minted}

Por otro lado, si el número de chequeos que se preveen tener es bajo, es posible
reducir el número de workers que utiliza \textbf{Celery} modificando el fichero
\texttt{supervisor-siteup.conf} que se generó en los pasos anteriores, cambiando
el valor del parámetro \texttt{-c 16} a un número menor.


%%% Local Variables: 
%%% mode: latex
%%% TeX-master: "../memoria"
%%% End: 
