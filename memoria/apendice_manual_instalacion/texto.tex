A continuación se presentan las instrucciones de instalación de la plataforma
web en un servidor de producción. La versión más actualizada de estas
instrucciones se pueden encontrar en el fichero \texttt{INSTALLING-web.md}
alojado en el directorio raíz de la forja de código~\cite{forja}.

Antes de continuar se presentan los requisitos de hardware y software mínimos
para la correcta ejecución de la plataforma web.

\paragraph{Hardware}

\begin{itemize}
\item 512MiB de memoria RAM como mínimo.
\item 10GiB de disco duro como mínimo.
\item Acceso a Internet con un canal de subida de al menos 1Mbit/s.
\end{itemize}

\paragraph{Software}

\begin{itemize}
\item Sistema operativo \textbf{GNU/Linux}, preferiblemente basado en paquetería Debian.
\item Servidor de shell remota \textbf{SSH}, accesible desde el exterior.
\item \textbf{Nginx}, servidor web trabajando en modo de proxy inverso.
\item \textbf{Supervisord}, sistema para el control de procesos que trabajen en modo demonio..
\item \textbf{RabbitMQ}, cola de tareas sencilla que cumple el estandar \ac{AMQP}.
\item Intérprete de \textbf{Python}, versión mínima 2.7.
\item Soporte de entornos virtuales \textbf{VirtualEnv} para la encapsulación de
  dependencias.
\end{itemize}


\section{Instalación inicial}

\subsection{Descarga de código}

Para instalar la plataforma web es necesario clonar el repositorio~\cite{forja}
con el código de la aplicación. Primero, creamos una ubicación donde alojar el
código desde la terminal:

\begin{bashcode}
$ mkdir /srv/siteup -p
$ cd /srv/siteup
\end{bashcode}

Una vez ahí, instalamos el software de control de versiones \textbf{git} y
clonamos el repositorio desde Github:

\begin{bashcode}
$ sudo apt-get install git
$ git clone https://github.com/JoseTomasTocino/pfc-ii.git .  
\end{bashcode}

\subsection{Entorno virtual y dependencias}

Como se ha comentado previamente, el proyecto utiliza \textbf{virtualenv} y
\textbf{virtualenvwrapper} para una gestión más limpia de las dependencias. La
instalación de estos dos elementos es sencilla, en primer lugar se instala el
gestor de paquetes y los dos paquetes mencionados:

\begin{bashcode}
$ sudo apt-get install python-pip
$ sudo pip install virtualenv
$ sudo pip install virtualenvwrapper
\end{bashcode}

En segundo lugar, añadimos el código de \textit{bootstrapping} de
virtualenvwrapper a nuestro perfil de terminal, habitualmente \texttt{.bashrc}:

\begin{bashcode}
$ cat >> ~/.bashrc

if [ -f /usr/local/bin/virtualenvwrapper.sh ]
then
    source /usr/local/bin/virtualenvwrapper.sh
fi

EOF
\end{bashcode}

Hecho esto, será necesario reiniciar la terminal, tras lo cual podremos crear el
entorno virtual e instalar las dependencias:

\begin{bashcode}
$ mkvirtualenv siteup
$ sudo apt-get install python-dev
$ pip install -r web/requirements.txt 
\end{bashcode}

El proceso de instalación de dependencias puede tardar entre 5 y 10 minutos aproximadamente.

\subsection{Instalación de credenciales}

Antes de continuar es necesario dar valor a algunas variables de credenciales
privadas que el proyecto necesita. En particular, se deben definir dos variables:

\begin{itemize}
\item \texttt{GMAIL\_PASS}, es la clave de la cuenta de correo utilizada para
  enviar notificaciones. Se define en el fichero \texttt{web/siteup/settings/base.py}.
\item \texttt{GCM\_API\_KEY}, es la clave de la API del servicio Google Cloud Messaging.
\end{itemize}

Para el proceso de instalación y en tiempo de desarrollo lo habitual es
añadirlas \textbf{script de activación} del entorno virtual, de la siguiente
manera:

\begin{bashcode}
$ cdvirtualenv
$ cat >> bin/postactivate
export GMAIL_PASS=yourpass
export GCM_API_KEY=yourkey

EOF
\end{bashcode}

Para la puesta en producción, esas variables se configurarán un poco más adelante.

\subsection{Creación de la base de datos}

El siguiente paso es preparar la base de adtos. Volvemos al directorio en el que
estábamos antes y ejecutamos el siguiente comando, que genera la estructura
básica de la base de datos para las tablas y aplica las migraciones existentes:

\begin{bashcode}
$ cd /srv/siteup
$ python web/manage.py syncdb
$ python web/manage.py migrate 
\end{bashcode}

Tras esto, la base de datos se habrá generado y se encontrará en el fichero\\
\texttt{web/db/django-db.sqlite3}.

\subsection{Disposición de ficheros estáticos}

Es importante que los ficheros estáticos de todos los componentes de la
aplicación se encuentren en un mismo lugar. Para ello, Django provee un comando
que unifica todos los estáticos en la carpeta indicada.

\begin{verbatim}
$ python web/manage.py collectstatic --noinput
\end{verbatim}



\subsection{Instalación del servidor}

En estas instrucciones vamos a usar \textbf{Nginx} como servidor frontal, aunque
también es posible utilizar Apache. Para instalar nginx, en sistemas GNU/Linux
basados en paquetería Debian, el procedimiento es muy sencillo:

\begin{bashcode}
$ sudo apt-get install nginx -y
\end{bashcode}

\subsection{Instalación del bróker de mensajes}

El bróker de mensajes elegido es \textbf{RabbitMQ}, aunque en principio valdría
cualquier otro bróker que fuese compatible con el estándar \ac{AMQP}. La
instalación es sencilla:

\begin{bashcode}
$ sudo apt-get install rabbitmq-server -y
\end{bashcode}

La configuración por defecto es suficiente para las necesidades del proyecto.

\subsection{Ejecución de los servicios}
\label{subsec:ejecucion_servicios}

Para el control de los servicios que ejecutan el proyecto es necesario utilizar
alguna clase de software de monitorización de procesos. La opción recomendada es
\textbf{supervisor}. Su instalación es sencilla:

\begin{bashcode}
$ sudo apt-get install supervisor -y
\end{bashcode}

Tras esto, es necesario generar el fichero de configuración de
supervisor. SiteUp integra una tarea que automáticamente genera este fichero por
nosotros. Solo tendremos que indicar las dos variables de credenciales que se
han comentado previamente, simplemente introduciendo su valor por teclado cuando
nos sea preguntado.

Para generar el fichero de configuración utilizaremos el siguiente comando:

\begin{bashcode}
$ python web/manage.py supervisor_conf
\end{bashcode}

Se generará un fichero \texttt{supervisor-siteup.conf} con la configuración. Es
buena idea revisar el fichero para comprobar que todas las rutas son
correctas. Tras ello, solo queda mover el fichero al lugar correcto y reiniciar
el demonio de supervisor.

\begin{bashcode}
$ sudo mv web/supervisor-siteup.conf /etc/supervisor/conf.d  
\end{bashcode}

\subsection{Configuración del servidor frontal}

El siguiente paso es configurar el servidor nginx para que reciba las peticiones
apropiadas. Vamos a suponer que SiteUp va a recibir las peticiones en el dominio
\texttt{siteup.uca.es}. Creamos el fichero
\texttt{/etc/nginx/sites-available/siteup} y añadimos el siguiente contenido:

\begin{minted}{nginx}
 server {
    listen 80;
    server_name siteup.uca.es;

    # Static files
    location /static/ {
        root /srv/siteup/web;
        try_files $uri $uri/ =404;
    }

    location / {
        proxy_set_header X-Real-IP        $remote_addr;
        proxy_set_header X-Forwarded-For  $proxy_add_x_forwarded_for;
        proxy_set_header Host             $http_host;
        proxy_redirect off;

        if (!-f $request_filename) {
            proxy_pass       http://127.0.0.1:8000;
            break;
        }
    }
} 
\end{minted}

Tras eso, activamos el fichero de configuración de la siguiente manera:

\begin{bashcode}
$ cd /etc/nginx/sites-enabled
$ sudo ln -s ../sites-available/siteup
\end{bashcode}

Básicamente lo que hace ese fichero de configuración es revisar las peticiones
que llegan y, si apuntan a la ruta \texttt{/static}, nginx se encarga de servir
esos ficheros estáticos de front-end. Si no, pasa la petición al servidor
gunicorn que estará escuchando en la dirección local \texttt{127.0.0.1:8000}.

\subsection{Lanzamiento final}

Tras toda esta configuración, los últimos pasos que quedan son activar los
servicios y servidores. Primero, lanzamos el controlador supervisor, que a su
vez lanzará el servidor dinámico gunicorn y el servidor de tareas celery:

\begin{bashcode}
$ sudo service supervisor restart  
\end{bashcode}

Seguidamente, reiniciamos el servidor frontal para que pueda empezar a recibir
peticiones:

\begin{bashcode}
$ sudo service nginx restart  
\end{bashcode}

Con esto, el proyecto estará funcionando y debería ser accesible a través del
navegador.

\section{Pruebas de implantación}

Para probar que la instalación es correcta, lo ideal es que pruebe la aplicación
a fondo, revisando toda la funcionalidad. El itinerario que se recomienda es el
siguiente. Si tiene dudas sobre cómo realizar alguna de las acciones descritas,
revise el~\nameref{chap:manual_usuario}.

El primer paso antes de nada es lanzar la batería de tests que viene incluida en
el proyecto, mediante el comando

\begin{verbatim}
$ python web/manage.py test
\end{verbatim}

Todos los tests deberían funcionar correctamente. Si no lo hacen, es que ha
habido un fallo en el sistema o en la aplicación.

Una vez descartados esos problemas, pruebe que el \textbf{registro y acceso de
  usuarios} funciona creando un nuevo usuario y accediendo a su cuenta. Si este
procedimiento funciona, se habrá verificado que la instalación básica de la base
de datos es correcta.

El siguiente paso es \textbf{crear un grupo de chequeos} y posteriormente crear
un chequeo de tipo Ping básico. Se recomienda utilizar alguna web conocida como
objetivo, por ejemplo la web de la Universidad de Cádiz en su dirección
\texttt{uca.es}.

Tras crear el chequeo, éste aparecerá en modo \textbf{Fetching}, durante el
que se están obteniendo los datos iniciales. Espere dos o tres minutos y vuelva
a revisar su estado. Si sigue en modo \textit{Fetching}, probablemente las
tareas de chequeo no se estén llevando a cabo, lo que seguramente sea causado
por una mala instalación del gestor de tareas \textbf{Celery} o de la cola de
mensajes \textbf{RabbitMQ}. 

Si el chequeo pasa a estado \textbf{Correcto}, entonces se habrá verificado el
correcto funcionamiento del gestor de tareas y de la cola de mensajes. El
siguiente paso es \textbf{comprobar las notificaciones}. Para ello, edite el
chequeo recién creado y cambie el objetivo (\textit{target}) a uno incorrecto,
por ejemplo \texttt{uca.foobar} y espere tres minutos. En ese periodo de tiempo
debería llegarle una notificación a través del correo electrónico. Si no es así,
probablemente la causa esté en que no esta bien configurada la cuenta de correo
electrónico para el envío de notificaciones. Revise para ello la sección
\textit{\nameref{subsec:ejecucion_servicios}}.

Si ha recibido correctamente la notificación a través del correo electrónico,
entonces el sistema de mailing estará funcionando correctamente. El siguiente
paso es repetir el proceso de \textbf{notificación con la aplicación
  Android}. Deberá instalar la aplicación en su móvil y acceder a la
misma. Seguidamente, desde la aplicación web deberá editar de nuevo el chequeo y
restaurar el objetivo original (\texttt{uca.es}), además de activar la
notificación a través de Android.

Tras dos o tres minutos el sistema debería descubrir el cambio de estado y
enviar las notificaciones, esta vez tanto por correo electrónico como por la
aplicación Android. Si no recibe la notificación en su dispositivo móvil puede
que haya algún problema en la configuración de las claves para Google Cloud
Messaging. De nuevo, debería revisar la sección
\textit{\nameref{subsec:ejecucion_servicios}} del manual de instalación.

Si ha recibido la notificación correctamente, puede dar por terminada y
\textbf{correcta} la verificación de la instalación.

\section{Problemas habituales}

\subsection{Activación de nginx}

A menudo, al instalar la configuración de nginx y reiniciar el servicio aparece el siguiente error:

\begin{verbatim}
nginx: [emerg] could not build the server_names_hash, you should increase server_names_hash_bucket_size: 32
\end{verbatim}

Al parecer, nginx está configurado por defecto para usar subdominios de 32
caracteres como máximo. Si su dominio tiene más caracteres, deberá editar el
fichero \texttt{/etc/nginx/nginx.conf} y descomentar la siguiente línea dentro
del bloque \texttt{http}:

\begin{verbatim}
server_names_hash_bucket_size 64;
\end{verbatim}

Tras eso, reinicie el servicio y el problema debería desaparecer.



\subsection{Mejora del rendimiento}

La configuración por defecto de \textbf{RabbitMQ} hace que consuma mucha memoria
y puede llegar a resultar un problema en sistemas con recursos limitados, como
los VPS de bajo coste.

Para arreglar esto, una opción es modificar el fichero de configuración,
situadio en \texttt{/etc/rabbitmq/rabbitmq.config} para limitar la memoria
máxima usada por Rabbit a un 15\% del total. El contenido a añadir al fichero es
el siguiente:

\begin{minted}{text}
[
    {rabbit, [{vm_memory_high_watermark, 0.15}]}
].  
\end{minted}

Por otro lado, si el número de chequeos que se preveen tener es bajo, es posible
reducir el número de workers que utiliza \textbf{Celery} modificando el fichero
\texttt{supervisor-siteup.conf} que se generó en los pasos anteriores, cambiando
el valor del parámetro \texttt{-c 16} a un número menor.


%%% Local Variables: 
%%% mode: latex
%%% TeX-master: "../memoria"
%%% End: 
