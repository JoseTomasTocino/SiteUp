Durante el transcurso del desarrollo de SiteUp, y sobre todo al término del
mismo, se han obtenido unas conclusiones y unos resultados, tanto de forma
personal como para con la comunidad, que intentaremos reflejar en este capítulo.


\section{Objetivos cumplidos}
Al término del desarrollo del proyecto, el proyecto ha completado todos los
objetivos a cumplir, presentes en el planteamiento inicial y detallados en la
sección \textit{\ref{sec:objetivos}}. En particular:

\begin{itemize}
\item Se ha creado un conjunto de herramientas de chequeo de servicios de
  Internet, que se encuentra disponible en el módulo
  \texttt{siteup\_checker/monitoring} del proyecto.
\item Se ha creado una aplicación online de acceso público con la que los
  usuarios pueden gestionar sus chequeos de manera sencilla, que en la fecha de
  escritura de la presente memoria es accesible en la url
  \url{http://siteup.josetomastocino.com}.
\item Se ha establecido un sistema de notificaciones con el que mantener
  informados a los usuarios tanto por correo electrónico como mediante una
  aplicación móvil para el sistema operativo Android desarrollada a tal
  efecto. Actualmente la aplicación se encuentra disponible en la forja de
  código~\cite{forja} y en un futuro lo estará en el Google Play Store.
\end{itemize}

La compleción total de los objetivos funcionales también pone de manifiesto que
se han alcanzado satisfactoriamente los objetivos transversales y personales
dispuestos en el inicio. Se han afianzado fuertemente los conocimientos de
desarrollo web en general y con Django en particular hasta un nivel de
competencia suficiente para servir como baza en la búsqueda laboral. Además,
SiteUp ha servido satisfactoriamente como primera incursión en el mundo del
desarrollo de aplicaciones Android.

\subsection{Limitaciones del proyecto}
Aunque cubre una gran parte de los puntos de vigilancia habituales, la
aplicación se limita a ofrecer chequeo de respuesta de ping, chequeo de puertos,
chequeo de registros DNS y chequeo de cabeceras y contenidos HTTP.

La aplicación de Android no cuenta con ninguna funcionalidad para gestionar los
chequeos de un usuario, sino que sirve para recibir notificaciones instantáneas
provenientes de la aplicación web. Ésta, por otro lado, está completamente
adaptada para su uso a través de dispositivos móviles gracias al uso del
\textit{responsive web design}.

Idealmente los chequeos deberían hacerse simultáneamente desde diferentes
máquinas colocadas en diversos puntos geográficos, para así tener unos
resultados más fiables. La falta de infraestructuras y la finalidad didáctica
del proyecto limitan la aplicación a una estructura monolítica en la que los
chequeos se hacen desde una sola máquina, la misma que sirve el servicio web.

\section{Conclusiones personales}

SiteUp es uno de los pocos proyectos personales que han surgido para suplir una
\textbf{necesidad} real y específica y que, al término del desarrollo, han
conseguido su objetivo. La idea inicial, expuesta en la sección
\ref{sec:situacion-actual}, ha servido como eje motor del proyecto en todo
momento. Si se diese la situación presentada en el inicio, el proyecto SiteUp
sería capaz de detectar el problema e informar en consecuencia. 

\subsection{Lecciones aprendidas}

La cantidad de tecnologías diferentes utilizadas en cada uno de los niveles de
abstracción del proyecto es bastante grande. Adquirir una competencia básica en
todas estas tecnologías es una tarea de una envergadura importante, y la
correcta ejecución del sistema es comparable al mecanismo de un reloj, en el que
todas las piezas deben funcionar de manera coordinada.

Desde la configuración de los numerosos servicios usados en el entorno de
producción, al despliegue de la aplicación, pasando por el desarrollo de los
módulos de chequeo, hasta llegar a la implementación del código de la capa de
presentación, sin olvidar la implementación de la aplicación Android. En todos
los niveles, en unos en mayor medida que en otros, se ha alcanzado una soltura
suficiente para el desarrollo y posterior lanzamiento de un producto completo.

Al tratarse de un proyecto de ejecución continua en el que constantemente se
están haciendo chequeos y enviando notificaciones, el flujo de \textit{feedback}
entre los usuarios y el desarrollador (que en las etapas iniciales son la misma
persona) es mucho más \textbf{dinámico} que en cualquier otro proyecto en el que
la ejecución fuese momentánea, como un juego o una aplicación de
escritorio. Este planteamiento ha permitido que en SiteUp se hayan detectado
rápidamente conflictos y problemas en el sitio, como por ejemplo que algunos
usuarios prefieren no recibir resúmenes diarios de sus chequeos, motivando la
inclusión de una opción en sus perfiles para desactivar esta funcionalidad.

Finalmente, tras la conclusión del proyecto, se ha conseguido un \textbf{punto
  de vista} diferente de las alternativas presentadas en la sección
\ref{sec:estado-del-arte}, \textit{\nameref{sec:estado-del-arte}}. Con un
conocimiento más formado de lo que hay detrás (salvando las distancias) de estos
proyectos resulta más sencillo \textbf{evaluar} en qué aciertan y en qué fallan,
qué les falta y qué les sobra, y sobre todo qué se puede aprender de ellos para
su integración en SiteUp. En particular, me ha sorprendido que no sea común que
esta clase de servicios proporcionen una aplicación móvil nativa que sirva, al
menos, para la recepción de notificaciones, tal y como se hace en SiteUp. La
inversión es despreciable si la comparamos con la utilidad que provee.


\section{Trabajo futuro}

Durante la etapa de desarrollo y, sobre todo, durante las sesiones de pruebas
han surgido muchos frentes de trabajo futuro, unos de mayor relevancia que
otros, pero importantes al fin y al cabo y que aquí se presentan.

\subsection{SiteUp Probes}

\textbf{SiteUp Probes} es un proyecto que busca crear \textit{sondas} que
ejecuten de manera independiente los chequeos dados de alta en una instalación
de SiteUp, enviando los resultados de vuelta al servidor central. La idea es que
estas sondas esté geográficamente distribuidas, de forma que los resultados de
los chequeos no dependan tanto de la situación del nodo central, sino que se
calculen como una media de los resultados obtenidos por distintas sondas.

La motivación de esta ampliación viene dada por lo descrito en la sección
\ref{subsec:riesgo-compromiso-plataforma},
\textit{\nameref{subsec:riesgo-compromiso-plataforma}}. El uso de un solo punto
a la hora de lanzar los chequeos puede comprometer la veracidad y corrección de
los resultados.

Se trata de un escalón importante y complejo en el trabajo de ampliación del
proyecto. A nivel de implementación implicaría habilitar alguna clase de canal
de comunicación entre el nodo central de SiteUp y las sondas para la
comunicación de los chequeos a ejecutar y de los resultados de éstos. A nivel
logístico conllevaría la búsqueda de nuevos sistemas en los que instalar las
sondas, más allá del servidor de producción que está alquilado para el proyecto.

\subsection{Monetización}

Para asegurar la supervivencia de cualquier proyecto es necesario contar alguna
clase de \textbf{modelo de negocio} que sufrague los gastos que se
generan~\cite{startupsFail}. En el caso de SiteUp, la \textbf{monetización} se
puede conseguir mediante la creación de niveles de cuentas de usuarios, que
gocen de distintos beneficios. Por ejemplo:

\begin{itemize}
\item Las cuentas gratuitas solo podrán crear un chequeo, y el tiempo entre
  chequeos será como mínimo de 5 minutos.
\item Las cuentas básicas costarán 5€ al mes, podrán crear hasta 5 chequeos y el
  tiempo mínimo entre chequeos será de 2 minutos.
\item Las cuentas premium costarán 10€ al mes, podrán crear hasta 30 chequeos y
  el tiempo mínimo entre chequeos será de 1 minuto.
\end{itemize}

La implementación de este modelo de monetización no es compleja, aunque la
gestión de pasarelas de pago y cobros recurrentes es una de los procedimientos
más problemáticos a nivel de programación, sobre todo a la hora de verificar la
correcta ejecución del sistema. 

\subsection{Ampliación de los tipos de chequeos}

Los cuatro tipos de chequeos actualmente presentes en SiteUp cubren un gran
abanico de frentes a la hora de vigilar el estado de un servicio de Internet,
pero existen (y seguirán surgiendo) nuevos puntos que convendría comprobar. Como
ejemplos de posibles chequeos a añadir en un futuro se podrían considerar:

\begin{itemize}
\item Chequeo de servidores de bases de datos mediante el envío de consultas de
  verificación.
\item Chequeo de transacciones. Esto es, chequeos compuestos de varios pasos.
\item Chequeos de APIs con campos adicionales, como autenticación, cabeceras
  especiales o parámetros GET/POST.
\end{itemize}

El trabajo en este punto pasaría por codificar los nuevos modelos de forma
similar a los que ya existen, que heredan de un modelo base que hace sencilla la
extensión del sistema.

\subsection{Ampliación de tipos de notificación}

Actualmente, SiteUp envía notificaciones mediante correo electrónico y mediante
la aplicación Android. Como posibles ampliaciones se podrían considerar el uso
de mensajes cortos (SMS) y el desarrollo de una app para iOS capaz de recibir
notificaciones PUSH.

%%% Local Variables: 
%%% mode: latex
%%% TeX-master: "../memoria"
%%% End: 
