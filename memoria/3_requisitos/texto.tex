
En este capítulo procederemos a describir de manera formal los objetivos que se
presentaron de manera genérica en el apartado~\ref{sec:objetivos},
\textit{\nameref{sec:objetivos}}. 

En particular, se detallan:

\begin{itemize}
\item Los \textbf{objetivos del sistema}, que resumen a muy grandes rasgos la
  funcionalidad del proyecto.
\item Los \textbf{requisitos funcionales}, que definen las funciones del sistema
  software y sus componentes.
\item Los \textbf{requisitos de información}, que detallan los datos que el
  sistema usará durante el desarrollo y ejecución para llevar a cabo de forma
  adecuada su funcionalidad.
\item Los \textbf{requisitos no funcionales}, que, en general, especifican
  criterios de diversas categorías que el software debe cumplir para garantizar
  un buen nivel de calidad más allá de su funcionamiento.
\end{itemize}

Para terminar el capítulo se presentan diversas alternativas tecnológicas para
cubrir las necesidades surgidas en los requerimientos del sistema, detallando
las decisiones tomadas.


\section{Objetivos del sistema}

\renewcommand{\arraystretch}{1.5}

Los objetivos principales de SiteUp, definidos a alto nivel son los dos siguientes:

\begin{center}
  
  \begin{tabularx}{\textwidth}{|c|X|}
    \hline
    Título & Crear chequeos de varios tipos sobre servicios de internet \\

    \hline

    Descripción & SiteUp permitirá crear chequeos de diversas clases para la
    comprobación de servicios de internet, utilizando varias tecnologías de
    chequeo y múltiples vectores de verificación: ping, puertos, hipertexto y
    registros DNS. \\    

    \hline
  \end{tabularx}
  \captionof{table}{OBJ-1}
\end{center}


\begin{center}
  
  \begin{tabularx}{\textwidth}{|c|X|}
    \hline
    Título & Notificación a los usuarios \\

    \hline

    Descripción & SiteUp deberá notificar a sus usuarios cuando el estado de sus
    chequeos cambie. Estas notificaciones se harán tanto por móvil como a través
    de una aplicación Android. \\

    \hline
  \end{tabularx}
  \captionof{table}{OBJ-2}
\end{center}

\section{Catálogo de requisitos}

Se presentan a continuación los requisitos del sistema, tanto a nivel funcional como de información y no funcionales.

\subsection{Requisitos funcionales}

\subsubsection{Gestión de usuarios}

\begin{center}
  
  \begin{tabularx}{\textwidth}{|c|X|}
    \hline
    Título & Creación de cuenta de usuario \\

    \hline

    Descripción & El usuario de SiteUp deberá ser capaz de crear una cuenta de
    usuario para acceder a las funciones de SiteUp, proporcionando su nombre de
    usuario, dirección de correo electrónico y contraseña.\\

    \hline
  \end{tabularx}
  \captionof{table}{RQF-1}
\end{center}

\begin{center}
  
  \begin{tabularx}{\textwidth}{|c|X|}
    \hline
    Título & Edición de cuenta de usuario \\

    \hline

    Descripción & Un usuario logueado deberá ser capaz de editar sus datos
    personales y preferencias de usuario. En particular, deberá ser capaz de
    editar su nombre de usuario, dirección de correo electrónico y contraseña,
    además de opciones como la posibilidad de recibir un resumen diario del
    estado de sus chequeos.\\

    \hline
  \end{tabularx}
  \captionof{table}{RQF-2}
\end{center}

\FloatBarrier
\subsubsection{Gestión de grupos de chequeos}

\begin{center}
  
  \begin{tabularx}{\textwidth}{|c|X|}
    \hline
    Título & Creación de grupos de chequeos \\

    \hline

    Descripción & Un usuario logueado deberá ser capaz de crear un grupo de
    chequeos, indicando el título que identifique al grupo.\\

    \hline
  \end{tabularx}
  \captionof{table}{RQF-3}
\end{center}

\begin{center}
  
  \begin{tabularx}{\textwidth}{|c|X|}
    \hline
    Título & Edición de grupos de chequeos \\

    \hline

    Descripción & Un usuario logueado deberá ser capaz de editar un grupo de
    chequeos que previamente haya creado. En particular, deberá ser capaz de
    modificar el título del grupo.

    \\

    \hline
  \end{tabularx}
  \captionof{table}{RQF-4}
\end{center}

\begin{center}
  
  \begin{tabularx}{\textwidth}{|c|X|}
    \hline
    Título & Eliminación de grupos de chequeos \\

    \hline

    Descripción & Un usuario logueado deberá ser capaz de eliminar un grupo de
    chequeos, eliminando en el proceso tanto el grupo en sí como los chequeos
    que pertenezcan a ese grupo. \\

    \hline
  \end{tabularx}
  \captionof{table}{RQF-5}
\end{center}

\begin{center}
  
  \begin{tabularx}{\textwidth}{|c|X|}
    \hline
    Título & Activación y desactivación de grupos de chequeos \\

    \hline

    Descripción & Un usuario logueado deberá ser capaz de activar y desactivar
    un grupo de chequeos. En la práctica, esto activará o desactivará cada uno
    de los chequeos que pertenezcan al grupo de forma individual. \\

    \hline
  \end{tabularx}
  \captionof{table}{RQF-6}
\end{center}

\FloatBarrier
\subsubsection{Gestión de chequeos}

\begin{center}
  
  \begin{tabularx}{\textwidth}{|c|X|}
    \hline
    Título & Creación de chequeos \\

    \hline

    Descripción & Un usuario logueado deberá ser capaz de añadir a un grupo un
    chequeo que puede ser de cuatro tipos distintos: \textit{Ping check},
    \textit{Port check}, \textit{DNS Check} y \textit{HTTP Check}. Según el tipo
    de chequeo a dar de alta, el usuario deberá introducir una serie de
    detalles, siendo común a todos ellos el incluir un título que identifique el
    chequeo, el objetivo del chequeo, la periodicidad y las opciones de
    notificar mediante e-mail y Android.  Los detalles particulares de cada
    chequeo se definen en los
    requisitos de información. \\

    \hline
  \end{tabularx}
  \captionof{table}{RQF-7}
\end{center}


\begin{center}
  
  \begin{tabularx}{\textwidth}{|c|X|}
    \hline
    Título & Actualización de chequeos \\

    \hline

    Descripción & Un usuario logueado deberá ser capaz de actualizar un
    determinado chequeo, modificando cualquiera de los detalles que lo definen
    (a excepción del tipo de chequeo, que es fijo). \\

    \hline
  \end{tabularx}
  \captionof{table}{RQF-8}
\end{center}

\begin{center}
  
  \begin{tabularx}{\textwidth}{|c|X|}
    \hline
    Título & Eliminación de chequeos \\

    \hline

    Descripción & Un usuario logueado deberá ser capaz de eliminar un
    determinado chequeo, evitando así que el sistema lo tenga en cuenta a la
    hora de lanzar las monitorizaciones. \\

    \hline
  \end{tabularx}
  \captionof{table}{RQF-9}
\end{center}

\begin{center}
  
  \begin{tabularx}{\textwidth}{|c|X|}
    \hline
    Título & Activación y desactivación de chequeos \\

    \hline

    Descripción & Un usuario logueado deberá ser capaz de activar o desactivar
    un chequeo, de forma que mientras esté activado, el chequeo será lanzado
    periódicamente por el motor de chequeos, y cuando esté desactivado el
    chequeo será ignorado por el sistema hasta que el usuario lo active de
    nuevo. \\

    \hline
  \end{tabularx}
  \captionof{table}{RQF-11}
\end{center}

\begin{center}
  
  \begin{tabularx}{\textwidth}{|c|X|}
    \hline
    Título & Revisión de chequeos \\

    \hline

    Descripción & Un usuario logueado deberá ser capaz de revisar los datos de
    monitorización que se hayan generado para un chequeo que previamente haya
    dado de alta, pudiendo ver una gráfica con el estado del chequeo a lo largo
    del tiempo. Además debe ser capaz de elegir el período de tiempo en el que
    revisar los datos (últimas 24 horas, última semana, etcétera). \\

    \hline
  \end{tabularx}
  \captionof{table}{RQF-11}
\end{center}


\begin{center}
  
  \begin{tabularx}{\textwidth}{|c|X|}
    \hline
    Título & Recepción de notificaciones por correo electrónico \\

    \hline

    Descripción & Un usuario logueado deberá ser capaz de recibir notificaciones
    a través del correo electrónico cuando el estado de un chequeo cambie,
    siempre que el usuario haya activado las notificaciones por correo para ese
    chequeo. \\

    \hline
  \end{tabularx}
  \captionof{table}{RQF-12}
\end{center}

\FloatBarrier
\subsubsection{Aplicación Android}


\begin{center}
  
  \begin{tabularx}{\textwidth}{|c|X|}
    \hline
    Título & Listado de chequeos en la aplicación Android \\

    \hline

    Descripción & Un usuario deberá ser capaz de ejecutar la aplicación de
    Android, loguearse con su cuenta de usuario y ver un listado de los chequeos
    que ha dado de alta en el sistema, así como un resumen de su estado.
    \\

    \hline
  \end{tabularx}
  \captionof{table}{RQF-13}
\end{center}


\begin{center}
  
  \begin{tabularx}{\textwidth}{|c|X|}
    \hline
    Título & Recepción de notificaciones por la aplicación Android \\

    \hline

    Descripción & Un usuario logueado deberá ser capaz de recibir notificaciones
    a través de la aplicación Android cuando el estado de un chequeo cambie,
    siempre que el usuario haya activado las notificaciones para ese
    chequeo. \\

    \hline
  \end{tabularx}
  \captionof{table}{RQF-14}
\end{center}

\FloatBarrier
\subsection{Requisitos de información}
\label{sec:requisitos-informacion}


\begin{center}
  
  \begin{tabularx}{\textwidth}{|l|X|}
    \hline

    & Usuario del sistema\\

    \hline
    Datos específicos &

    \begin{itemize}
    \item Nombre de usuario
    \item Dirección de correo electrónico
    \item Contraseña
    \item (\textit{Opcional}) Identificador de dispositivo Android
    \end{itemize}
    \\
    
    \hline
    
  \end{tabularx}
  \captionof{table}{IRQ-1}
\end{center}


\begin{center}
  
  \begin{tabularx}{\textwidth}{|l|X|}
    \hline

    & Chequeo tipo Ping \\

    \hline
    Datos específicos &

    \begin{itemize}
      \item Título 
      \item Descripción
      \item Objetivo -- Debe ser una IP o un hostname
      \item Frecuencia de chequeo en minutos
      \item Notificar por correo
      \item Notificar por Android
      \item Si se debe chequear el tiempo de respuesta
      \item Tiempo máximo de respuesta en milisegundos
    \end{itemize}
    \\
    
    \hline
    
  \end{tabularx}
  \captionof{table}{IRQ-2}
\end{center}

\begin{center}
  
  \begin{tabularx}{\textwidth}{|l|X|}
    \hline

    & Chequeo tipo DNS \\

    \hline
    Datos específicos &

    \begin{itemize}
      \item Título 
      \item Descripción
      \item Objetivo -- Debe ser un hostname
      \item Tipo de registro DNS (A, AAAA, CNAME, MX, TXT)
      \item Contenido esperado del registro DNS
      \item Frecuencia de chequeo en minutos
      \item Notificar por correo
      \item Notificar por Android
    \end{itemize}
    \\
    
    \hline
    
  \end{tabularx}
  \captionof{table}{IRQ-3}
\end{center}

\begin{center}
  
  \begin{tabularx}{\textwidth}{|l|X|}
    \hline

    & Chequeo tipo Port \\

    \hline
    Datos específicos &

    \begin{itemize}
      \item Título 
      \item Descripción
      \item Objetivo -- Debe ser una IP o un hostname
      \item Puerto objetivo
      \item \textit{Opcional} - Cadena de caracteres que debe aparecer en la respuesta
      \item Frecuencia de chequeo en minutos
      \item Notificar por correo
      \item Notificar por Android
    \end{itemize}
    \\
    
    \hline
    
  \end{tabularx}
  \captionof{table}{IRQ-4}
\end{center}

\begin{center}
  
  \begin{tabularx}{\textwidth}{|l|X|}
    \hline

    & Chequeo tipo HTTP \\

    \hline
    Datos específicos &

    \begin{itemize}
      \item Título 
      \item Descripción
      \item Objetivo -- Debe ser una \ac{URL} \ac{HTTP}
      \item Código de estado HTTP
      \item \textit{Opcional} - Cadena de caracteres que debe aparecer en la respuesta
      \item Frecuencia de chequeo en minutos
      \item Notificar por correo
      \item Notificar por Android
    \end{itemize}
    \\
    
    \hline
    
  \end{tabularx}
  \captionof{table}{IRQ-5}
\end{center}

\begin{center}
  
  \begin{tabularx}{\textwidth}{|l|X|}
    \hline

    & Registro de un chequeo \\

    \hline
    Datos específicos &

    \begin{itemize}
    \item Fecha y hora de registro
    \item Estado (Up, Down, Error)
    \item Tiempo de respuesta (solo para chequeos tipo Ping).
    \item Información adicional
    \item Chequeo asociado
    \end{itemize}
    \\
    
    \hline
    
  \end{tabularx}
  \captionof{table}{IRQ-6}
\end{center}

\begin{center}
  
  \begin{tabularx}{\textwidth}{|l|X|}
    \hline

    & Estado de un chequeo \\

    \hline
    Datos específicos &

    \begin{itemize}
    \item Fecha y hora de inicio del estado
    \item Fecha y hora de fin del estado
    \item Estado (Up, Down, Error)
    \item Información adicional
    \item Chequeo asociado
    \end{itemize}
    \\
    
    \hline
    
  \end{tabularx}
  \captionof{table}{IRQ-7}
\end{center}

\begin{center}
  
  \begin{tabularx}{\textwidth}{|l|X|}
    \hline

    & Grupo de chequeos \\

    \hline
    Datos específicos &

    \begin{itemize}
      \item Título
      \item Dueño
    \end{itemize}
    \\
    
    \hline
    
  \end{tabularx}
  \captionof{table}{IRQ-8}
\end{center}

\FloatBarrier
\subsection{Requisitos no funcionales}

\subsubsection{Requisitos de seguridad}

\begin{center}
  
  \begin{tabularx}{\textwidth}{|c|X|}
    \hline
    Título & Cifrado de contraseñas de usuario \\

    \hline

    Descripción & Las contraseñas de los usuarios se almacenarán cifradas con un
    algoritmo de un solo sentido, de acuerdo a lo establecido en la vigente \ac{LOPD}. \\


    \hline
  \end{tabularx}
  \captionof{table}{NRQ-1}
\end{center}

\begin{center}
  
  \begin{tabularx}{\textwidth}{|c|X|}
    \hline
    Título & Restricciones de acceso a usuarios \\

    \hline

    Descripción & Los datos de los chequeos dados de alta sólo serán accesibles
    por los usuarios que los hayan creado y los administradores de la plataforma. \\


    \hline
  \end{tabularx}
  \captionof{table}{NRQ-2}
\end{center}

\FloatBarrier
\subsubsection{Requisitos de fiabilidad}


\begin{center}
  
  \begin{tabularx}{\textwidth}{|c|X|}
    \hline
    Título & Ejecución de chequeos \\

    \hline

    Descripción & Los retrasos en la periodicidad de los chequeos no puede
    superar el 200\% del tiempo entre chequeos. Esto es, si un chequeo tiene una
    periodicidad de 1 minuto, en caso de sobrecarga del sistema el tiempo máximo
    entre chequeos será de 3 minutos (un retraso del 200\%). \\

    \hline
  \end{tabularx}
  \captionof{table}{NRQ-3}
\end{center}


\FloatBarrier
\subsubsection{Requisitos de accesibilidad y usabilidad}

\begin{center}
  
  \begin{tabularx}{\textwidth}{|c|X|}
    \hline
    Título & Adaptación a dispositivos \\

    \hline

    Descripción & La plataforma web deberá ser accesible desde cualquier clase
    de dispositivo móvil con acceso a internet: móviles, tablets, etcétera,
    manteniendo siempre la accesibilidad y el grueso del contenido. \\

    \hline
  \end{tabularx}
  \label{tab:accesibilidad}
  \captionof{table}{NRQ-4}
\end{center}


\FloatBarrier
\subsubsection{Requisitos de eficiencia}

\begin{center}
  
  \begin{tabularx}{\textwidth}{|c|X|}
    \hline
    Título & Monitorización de peticiones de demasiada duración \\

    \hline

    Descripción & En aras de mantener el sistema fluyendo, el motor de chequeos
    deberá ser capaz de identificar y abortar los chequeos que, por causas
    propias o ajenas al sistema, estén tardando más de lo previsto e incidiendo
    negativamente en la eficiencia del sistema. \\

    \hline
  \end{tabularx}
  \captionof{table}{NRQ-5}
\end{center}

\section{Alternativas de solución}
\label{sec:alternativas-solucion}

En esta sección, se presentan diferentes alternativas tecnológicas que permiten
satisfacer los requerimientos del sistema, presentando las alternativas elegidas
y detallando las razones de esta decisión.


\subsection{Frameworks para el desarrollo de la plataforma web}

Dada la mayor familiaridad del desarrollador con el lenguaje, solo se barajaron
frameworks de desarrollo basados en el lenguaje Python.

\begin{itemize}
\item \textbf{Django}\cite{django}: popular framework para el desarrollo rápido
  de aplicaciones web. Consta de una arquitectura similar al popular patrón
  \ac{MVC}, y cuenta con una basta comunidad de usuarios. Los proyectos Django
  se organizan en \textit{apps}, que permiten modularizar la
  funcionalidad. Integra un \ac{ORM}, un sistema de plantillas y una sección de
  administración adaptable a cualquier sitio web.

\item \textbf{Flask}\cite{flask}: se trata de un minimalista framework para Python que
  provee las funcionalidades más básicas, dejando al desarrollador elegir el
  resto de componentes. En los últimos meses, Flask ha ganado gran popularidad
  frente a Django al dar más libertad al programador.

\item \textbf{Bottle}\cite{bottle}: similar a Flask, Bottle es otro
  micro-framework para Python, siendo tan minimalista que está encapsulado en un
  solo fichero fuente y no necesita de dependencias externas. Se encuentra en
  etapas tempranas de su desarrollo.

\end{itemize}

Se tomó la decisión de optar por \textbf{Django} por ser el framework más
robusto y con mayor número de utilidades. Su uso por grandes proyectos, como
Instagram, afianzó aún más la decisión de usarlo.

\subsection{Sistemas de gestión de bases de datos}

\begin{itemize}
\item \textbf{MySQL}\cite{mysql}, uno de los sistemas de bases de datos más populares. Se
  usa en multitud de proyectos de gran envergadura y su utilidad y fiabilidad
  están más que demostradas. Cuenta con una arquitectura cliente-servidor, y
  suele necesitar bastantes recursos.

\item \textbf{SQLite}\cite{sqlite}: potente pero ligero, SQLite se aleja de la arquitectura
  cliente-servidor y se integra directamente en la aplicación cliente pasando a
  ser parte integral del mismo. Esto reduce la latencia en las comunicaciones de
  datos. Es el SGBD que Django usa por defecto.

\end{itemize}

En este aspecto, se ha tomado la decisión de utilizar SQLite por varias
razones. En primer lugar, es la opción que usa Django por defecto. En segundo
lugar, los datos se guardan en un fichero, por lo que resulta trivial mover el
sistema de un servidor a otro, lo cual es un gran ahorro de tiempo durante el
desarrollo. Finalmente, SQLite provee el rendimiento necesario para las etapas
tempranas del proyecto.

De cualquier modo, en un futuro no se descarta pasar a un SGBD basado en la
arquitectura cliente-servidor, como MySQL.

\subsection{Sistemas para la ejecución asíncrona de tareas}

\begin{itemize}
\item \textbf{Cron}\cite{cron} es un demonio~\footnote{Entiéndase \textit{demonio} en el
    sentido de un servicio que se ejecuta en segundo plano.} encargado de lanzar
  procesos de forma periódica utilizado en sistemas operativos basados en
  Unix. Cada usuario en un sistema cuenta con un fichero \textit{crontab}, en el
  que puede definir qué tareas ejecutar y con qué periodicidad deben ejecutarse. 

\item \textbf{Celery}\cite{celery} es una cola de tareas asíncronas desarrollado en
  Python. Una instalación de celery cuenta con una serie de \textit{workers} que
  leen tareas de una cola de mensajes (basada en el estándar \ac{AMQP}) y las
  ejecutan de forma síncrona o asíncrona, opcionalmente guardando su resultado
  en alguna clase de estructura. 

\end{itemize}

De las opciones presentadas finalmente se decidió usar \textbf{Celery} por
varias razones:

\begin{itemize}
\item Facilidad al dar de alta nuevas tareas. En el caso de cron, es necesario
  gestionar manualmente las tareas a ejecutar y añadirm también de forma manual,
  el proceso que lance esas tareas al fichero \textit{crontab}. En el caso de
  celery, contamos con unas bibliotecas con las que simplemente damos de alta
  una tarea y el sistema se encarga de ponerla en cola y ejecutarla en el
  momento adecuado.

\item Monitorización. Una de las mayores dificultades que presenta cron es la
  monitorización de su ejecución. No existe una certeza absoluta de que se estén
  lanzando los procesos, al menos que se elabore un sistema de verificación
  externo. En cambio, celery cuenta con sistemas de monitorización de la
  ejecución y \textit{logs} en los que se registra el progreso de cada tarea.

\item Integración. Al estar escrito en Python, celery se integra perfectamente
  con cualquier proyecto Django. Sin embargo, cron no ofrece de forma nativa
  ninguna opción de integración.

\end{itemize}

\subsection{Plataformas para el desarrollo de la aplicación móvil}

Como se especificó en la sección~\ref{sec:objetivos}, uno de los objetivos
principales del proyecto era el desarrollo, junto a la plataforma web, de una
aplicación móvil para que los usuarios pudiesen recibir notificaciones
instantáneas en cualquier momento y lugar, directamente en su móvil. Así pues,
resultó necesario hacer un estudio de las plataformas móviles existentes para
ver en cuál de ellas se centraría el desarrollo.

\begin{itemize}
\item \textbf{iOS}\cite{ios} es un sistema operativo móvil de la empresa
  \textbf{Apple} utilizado en los dispositivos de la familia iPhone, iPod Touch
  e iPad. A pesar de estar basado en un núcleo \ac{BSD} libre, iOS tiene una
  licencia privativa.

  El desarrollo para iOS es bastante estricto. Sólo es posible desarrollar para
  iOS utilizando computadoras Mac con el sistema operativo OS X 10.8 o
  superior\cite{ios-development-requirements}. La programación se lleva a cabo
  utilizando \textbf{Objective-C}, un lenguaje derivado del C cuyo uso a día de
  hoy prácticamente se limita al desarrollo en iOS. Además, es necesario
  utilizar el conjunto de herramientas que provee Apple, en particular XCode.

  El control de Apple sobre las aplicaciones que se desarrollan es férreo. La
  \textit{App Store}, aplicación desde la que los usuarios instalan nuevos
  programas, controla fuertemente las aplicaciones que se envían mediante
  exhaustivas revisiones, que suelen tardar entorno a una semana. Además, para
  poder lanzar una app es necesario pagar una cuota anual de 99 dólares.

\item \textbf{Android}\cite{android} es un sistema operativo móvil basado en el
  kernel de Linux, diseñado para su uso en móviles y tabletas. Inicialmente su
  desarrollo fue llevado a cabo por la empresa Android, Inc, pero posteriormente
  Google compró la empresa y es la que en la actualidad impulsa el desarrollo.

  El desarrollo en Android es bastante más laxo que en iOS. El lenguaje de
  programación utilizado es Java, un lenguaje compilado muy conocido y utilizado
  en muchos ámbitos, más allá del desarrollo móvil. Es posible desarrollar en
  cualquier plataforma (GNU/Linux, Windows o Mac), utilizando cualquier entorno
  de desarrollo que sea capaz de compilar código Java con las bibliotecas de
  Android.

  Las aplicaciones desarrolladas para Android se pueden lanzar en un mercado de
  aplicaciones llamado \textit{Google Play Store}, aunque también existen
  mercados de terceros al igual que es posible instalar aplicaciones de forma
  manual. Google cobra una tarifa única de 25 dólares por desarrollador, válida
  de por vida. Las revisiones que hace Google de las aplicaciones son menos
  rígidas que en el caso de Apple, y suelen tardar mucho menos.

\item \textbf{BlackBerry}, \textbf{Windows Phone} y \textbf{Symbian} son otras
  paltaformas móviles de mucho menor calado y público. El desarrollo para esas
  plataformas está mucho menos extendido, tiene menor soporte y menor clientela.

  BlackBerry OS es un sistema operativo que nació en 1999, época en la que
  BlackBerry era la compañía estrella en dispositivos \textit{handheld}, siendo
  la opción principal para empresarios y gente de negocios que tuviese necesidad
  de chequear el correo electrónico o notas estando lejos de una computadora. En
  la actualidad, el número de dispositivos que utilizan BlackBerry ha caído
  estrepitosamente, principalmente por la limitada compatibilidad del sistema
  BlackBerry 10 con otros dispositivos.

  Windows Phone es la apuesta de Microsoft en el mercado móvil. Lanzado en 2010,
  la última versión del sistema es Windows Phone 8.1. Actualmente se encuentra
  en pleno crecimiento, principalmente gracias al dominio de Nokia en mercados
  crecientes como Asia, América Latina y África\cite{wp-beats-ios}. A pesar de
  ello, todavía no es una plataforma firme con una base de usuarios a considerar.

  Symbian es una plataforma libre que tuvo un gran éxito en la década de 2000,
  siendo la alternativa más asequible a BlackBerry hasta la aparición de Android
  en 2010. La última versión del sistema se lanzó en octubre de 2012, y desde
  entonces no habido más desarrollo.

\end{itemize}

\subsubsection{Cuotas de mercado}

En la tabla~\ref{tab:cuota-2014} se detallan las cuotas de mercado de las
principales plataformas móviles medidas en enero de 2014 entre usuarios mayores
de 13 años de edad~\cite{comscore-report-2014}.

\begin{table}[hbtp]
  \centering
  \begin{tabular}[h]{|l|l|}
    \hline
    \textit{Plataforma} & \textit{Porcentaje} \\
    \hline
    Android & 51.7\% \\
    Apple & 41.6\% \\
    BlackBerry & 3.1\% \\
    Microsoft & 3.2\% \\
    Symbian & 0.2\% \\
    \hline
  \end{tabular}
  \caption{Cuota de mercado de plataformas móviles en enero de 2014}
  \label{tab:cuota-2014}
\end{table}

\subsubsection{Decisión}

Así, dadas las evidencias en la cuota de mercado, facilidad de desarrollo y
costes en licencias, se tomó la decisión de usar \textbf{Android} como
plataforma móvil principal a la hora de desarrollar la aplicación móvil.

En un futuro, según las circunstancias, se valorará trasladar el desarrollo
también a la plataforma iOS, aunque es algo harto improbable dadas las
exigencias de infraestructura (es necesario tener un ordenador Mac reciente) y
monetarias (el coste de la licencia anual es prohibitivo para una aplicación
gratuita).

%%% Local Variables: 
%%% mode: latex
%%% TeX-master: "../memoria"
%%% End: 
