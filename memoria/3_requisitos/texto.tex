\section{Estado del arte}

En la actualidad existen bastantes servicios web que ofrecen algunas
características similares a las que se desean en SiteUp, pero no todas. Se
presentan a continuación algunos de estos servicios, junto a los problemas que
se han detectado.

\begin{description}
\item[Pingdom] \url{http://pingdom.com} -- La opción más veterana y popular, con
  clientes muy importantes. La cuenta gratuita solo permite añadir un check. No
  ofrece chequeo de registros DNS.
\item[Alertra] \url{http://alertra.com} -- La cuenta gratuita solo dura 30
  días. No ofrece chequeo de registros DNS.
\item[UptimeRobot.com] \url{http://uptimerobot.com} -- Periodicidad mínima de 5
  minutos. No ofrece chequeo de DNS.
\item[ServerCheck] \url{http://servercheck.in} -- Aspecto amateur. No tienen
  cuenta gratuita. No tienen chequeo de DNS.
\item[StatusCake] \url{http://statuscake.com} -- Periodicidad mínima de 5
  minutos. No permite el chequeo de códigos de estado en su cuenta gratuita.

\end{description}

Una carencia importante que no se ha mencionado por ser generalizada es la falta
de aplicaciones nativas para móviles. Así pues, queda patente la dificultad de
encontrar un servicio que aúne el mayor número de características posibles a la
vez que mantiene un servicio gratuito y de calidad.

\subsection{Motivación personal}

En lo personal, la idea de desarrollar este proyecto surgió a raíz de una
necesidad personal. Durante un importante proceso de selección que estaba
gestionando por correo electrónico, la empresa que gestionaba el dominio de mi
web personal (y, por extensión, mi correo electrónico) tuvo un problema y
reemplazó los registros DNS personalizados por unos por defecto. Las alertas que
tenía puestas para verificar que mi web estaba operativa no dieron la alarma, ya
que el dominio ahora apuntaba a un espacio de párking genérico que no generaba
ningún código de error, pero los correos electrónicos no se estaban enrutando
correctamente. Esta situación se prolongó durante varios días, en los cuales
perdí varios correos importantes

Esto se podría haber detectado rápidamente si hubiera tenido un chequeo que
verificase que los registros DNS apuntaban a las IP correctas. A partir de ahí,
se hizo un estudio de mercado, cuyas conclusiones se han reflejado en la sección
anterior, y dadas las obvias limitaciones de las alternativas existentes se
decidió elaborar un proyecto nuevo.

\section{Objetivos del sistema}

Los objetivos principales de SiteUp son:

\begin{table}[h!]
  \centering
  \begin{tabularx}{\textwidth}{|c|X|}
    \hline
    Título & Crear chequeos de múltiples tipos sobre servicios de internet \\

    \hline

    Descripción & SiteUp permitirá crear chequeos de diversas clases para la comprobación
    de servicios de internet, utilizando varias tecnologías de chequeo y múltiples vectores de verificación: ping, puertos, hipertexto y registros DNS. \\    

    \hline
  \end{tabularx}
  \caption{OBJ-1}
\end{table}


\begin{table}[h!]
  \centering
  \begin{tabularx}{\textwidth}{|c|X|}
    \hline
    Título & Notificación a los usuarios \\

    \hline

    Descripción & SiteUp deberá notificar a sus usuarios cuando el estado de sus chequeos cambie. Estas notificaciones se harán tanto por móvil como a través de una aplicación Android. \\

    \hline
  \end{tabularx}
  \caption{OBJ-2}
\end{table}

\section{Catálogo de requisitos}

Se presentan a continuación los requisitos del sistema, tanto a nivel funcional como de información y no funcionales.

\subsection{Requisitos funcionales}

\subsubsection{Gestión de usuarios}

\begin{table}[h!]
  \centering
  \begin{tabularx}{\textwidth}{|c|X|}
    \hline
    Título & Creación de cuenta de usuario \\

    \hline

    Descripción & El usuario de SiteUp deberá ser capaz de crear una cuenta de usuario para acceder a las funciones de SiteUp, proporcionando su nombre de usuario,
                  dirección de correo electrónico y contraseña. \\

    \hline
  \end{tabularx}
  \caption{RQF-1}
\end{table}

\begin{table}[h!]
  \centering
  \begin{tabularx}{\textwidth}{|c|X|}
    \hline
    Título & Edición de cuenta de usuario \\

    \hline

    Descripción & Un usuario logueado deberá ser capaz de editar sus datos personales y preferencias de usuario. En particular, deberá ser capaz de editar su nombre de usuario,
dirección de correo electrónico y contraseña, además de opciones como la posibilidad de recibir un resumen diario del estado de sus chequeos. \\

    \hline
  \end{tabularx}
  \caption{RQF-2}
\end{table}

\FloatBarrier
\subsubsection{Gestión de chequeos}

\begin{table}[h!]
  \centering
  \begin{tabularx}{\textwidth}{|c|X|}
    \hline
    Título & Creación de chequeos \\

    \hline

    Descripción & Un usuario logueado deberá ser capaz de dar de alta chequeos
    de cuatro tipos distintos: \textit{Ping check}, \textit{Port check},
    \textit{DNS Check} y \textit{HTTP Check}. Según el tipo de chequeo a dar de
    alta, el usuario deberá introducir una serie de detalles, siendo común a
    todos ellos el incluir un título que identifique el chequeo, el objetivo del
    chequeo, la periodicidad y las opciones de notificar mediante e-mail y
    Android.  Los detalles particulares de cada chequeo se definen en los
    requisitos de información. \\

    \hline
  \end{tabularx}
  \caption{RQF-3}
\end{table}


\begin{table}[h!]
  \centering
  \begin{tabularx}{\textwidth}{|c|X|}
    \hline
    Título & Actualización de chequeos \\

    \hline

    Descripción & Un usuario logueado deberá ser capaz de actualizar un
    determinado chequeo, modificando cualquiera de los detalles que lo definen
    (a excepción del tipo de chequeo, que es fijo). \\

    \hline
  \end{tabularx}
  \caption{RQF-4}
\end{table}

\begin{table}[h!]
  \centering
  \begin{tabularx}{\textwidth}{|c|X|}
    \hline
    Título & Eliminación de chequeos \\

    \hline

    Descripción & Un usuario logueado deberá ser capaz de eliminar un
    determinado chequeo, evitando así que el sistema lo tenga en cuenta a la
    hora de lanzar las monitorizaciones. \\

    \hline
  \end{tabularx}
  \caption{RQF-5}
\end{table}





%%% Local Variables: 
%%% mode: latex
%%% TeX-master: "../memoria"
%%% End: 
