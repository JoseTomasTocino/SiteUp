
\section{Metodología}
\textbf{oFlute} ha seguido una metodología de desarrollo ágil en la que,
mediante fases de desarrollo rápidas y ligeras, se intenta evitar los formales
caminos de las metodologías tradicionales, enfocándose en las personas y
obteniendo así resultados en etapas más tempranas del desarrollo.

En las sucesivas secciones y capítulos se detallará el proceso de análisis y
posteriores fases del proyecto en la última iteración del proyecto. Así se
consigue una documentación más concisa y cercana al producto final.

\section{Especificación de requisitos del sistema}

\subsection{Requisitos de interfaces externas}
En esta sección describiremos los requisitos que deben cumplir las interfaces
con el hardware, el software y el usuario.

En cuanto a la comunicación con el subsistema gráfico y de E/S, utilizaremos la
biblioteca Gosu~\cite{gosu}, un proyecto de software libre que proporciona un
framework de desarrollo de videojuegos 2D, multiplataforma y muy sencillo de
usar. Para el acceso al subsistema de audio, tal y como se ha comentado en la
sección anterior, optamos por utilizar la API simple de PulseAudio.

\textbf{oFlute} dispondrá de una resolución fija de 800 por 600 píxeles,
requisito fácilmente alcanzable en cualquier ordenador actual. Al tratar con un
público objetivo joven, los gráficos y la interactividad deberán ser sencillos y
fáciles de interpretar. Así, se ha trabajado en limitar la interacción del
usuario con la aplicación al uso del ratón y, obviamente, del instrumento
musical, en este caso la flauta dulce. La navegación resultante de este
planteamiento queda reflejada en el siguiente diagrama:

\begin{figure}[h!]
  \centering
  \includegraphics[width=0.9\textwidth]{4_analisis/imagen_diagrama_de_flujo}
  \caption{Diagrama de flujo de las pantallas de oFlute}
\end{figure}

\pagebreak

Inicialmente, deberán aparecer unas pantallas de crédito con información sobre
el desarrollador y sobre el propio videojuego. Tras las mismas, que deberá ser
posible omitir, habrá de aparecer el \textbf{menú principal}, con las cinco
opciones posibles.

\begin{figure}[h!]
  \centering
  \includegraphics[width=0.6\textwidth]{4_analisis/imagen_mockup_menu_principal}
  \caption{Maqueta del menú principal}
\end{figure}

\pagebreak

Las opciones que se incluirán son:
\begin{itemize}
\item \textbf{Analizador de notas}: comprobar las notas que tocamos sobre un
  pentagrama.
\item \textbf{Canciones}: sección principal del juego, en el que aparecerán las
  canciones a tocar.
\item \textbf{Lecciones}: sección de lecciones de aprendizaje.
\item \textbf{Calibrar micrófono}, para ajustarse al nivel de ruido ambiental.
\item \textbf{Salir} al sistema operativo.
\end{itemize}

La siguiente pantalla a modelar será el \textbf{analizador de
  notas}. Simplemente mostrará el logotipo del videojuego a un lado, y un
pentagrama al otro, que se actualizará con la nota detectada por el
micrófono. También contendrá un botón \textit{volver} para ir al menú principal.

\begin{figure}[h!]
  \centering
  \includegraphics[width=0.6\textwidth]{4_analisis/imagen_mockup_analizador}
  \caption{Maqueta de la sección \textit{analizador de notas}}
\end{figure}

La segunda sección a la que se podrá ir desde el menú principal será la de
\textbf{canciones}. Inicialmente, la primera pantalla será la de
\textbf{selección de canción}, que contendrá el logotipo del juego, un botón
para volver al menú principal, y un menú dinámico de canciones que nos permitirá
elegir el tema a interpretar.

\begin{figure}[h!]
  \centering
  \includegraphics[width=0.6\textwidth]{4_analisis/imagen_mockup_seleccionar_cancion}
  \caption{Maqueta del menú de selección de canción}
\end{figure}

Una vez seleccionada la canción, pasaremos a la zona de \textbf{interpretación
  de canción}. Contendrá un pentagrama que ocupará todo el ancho de la pantalla,
con una línea que indicará la zona donde empezar a tocar las notas que
aparezcan. Además, en la parte superior habrá un indicador con la puntuación
obtenida y, abajo, una barra de progreso que nos indicará cuánto queda de
canción.

\begin{figure}[h!]
  \centering
  \includegraphics[width=0.6\textwidth]{4_analisis/imagen_mockup_reproducir_cancion}
  \caption{Maqueta de la pantalla de interpretación de canción}
\end{figure}

\pagebreak

Al completar la interpretación de la canción, aparecerá la \textbf{sección de
  resultados}. Contendrá el logotipo del juego, el título y subtítulo de la
canción, y un cuadro con información sobre nuestra interpretación, representada
en forma de porcentaje de aciertos. Además, en la zona inferior aparecerá un
mensaje de ánimo dependiendo del resultado obtenido.

\begin{figure}[h!]
  \centering
  \includegraphics[width=0.6\textwidth]{4_analisis/imagen_mockup_puntuaciones}
  \caption{Maqueta de la pantalla de puntuaciones}
\end{figure}

La pantalla de \textbf{selección de lecciones}, a la que se llega desde el menú
principal, contendrá el título, una imagen decorativa, y varios botones para
navegar entre las lecciones cargadas en el sistema. Se mostrará el título y la
descripción de cada lección, así como un botón para comenzar.

\begin{figure}[h!]
  \centering
  \includegraphics[width=0.6\textwidth]{4_analisis/imagen_mockup_seleccionar_leccion}
  \caption{Maqueta del menú de selección de lecciones}
\end{figure}

Una vez elegida una lección, pasaremos a la pantalla de reproducción de
lecciones. Dada la naturaleza \textbf{dinámica} de esta sección, cada lección
podrá tener una apariencia y elementos distintos. El único elemento común entre
todas las lecciones será el botón de \textbf{volver al menú}.

\subsection{Requisitos funcionales}

\textbf{oFlute} se basa en los siguientes requisitos funcionales:
\begin{itemize}
\item Poder terminar la aplicación pulsando el botón de cierre en cualquier
  instante.
\item Comprobar la correcta interpretación de notas individuales mediante el
  analizador de notas.
\item Calibrar el micrófono de forma que el sistema se pueda adaptar al ruido
  ambiental del entorno.
\item Navegar por toda la aplicación de forma sencilla utilizando solo el ratón.
\item Elegir entre varias canciones a interpretar, cada una con su título y
  subtítulo informativos.
\item Interpretar las canciones mediante el uso de la flauta, siguiendo el
  pentagrama en pantalla.
\item Elegir entre bastantes lecciones informativas, poder ejecutarlas y
  seguirlas.
\item Capacidad de añadir nuevas lecciones y canciones de forma sencilla.
\end{itemize}

\subsection{Requisitos de rendimiento}

La aplicación \textbf{oFlute} precisa de unos requisitos bastante básicos,
que en su mayor parte se reducen a cuatro puntos principales:
\begin{itemize}
\item Posesión de una tarjeta de sonido o subsistema de audio similar con un
  micrófono, para poder captar el sonido de la flauta.
\item Pantalla con una resolución de, al menos, 800 por 600 píxeles.
\item Sistema gráfico compatible con OpenGL.
\item Dispositivo apuntador, como un ratón.
\end{itemize}

La práctica totalidad de los ordenadores personales de la actualidad cumplen los
citados requisitos.

\subsection{Requisitos del sistema software}
El sistema de software deberá cumpliar los requisitos siguientes:
\begin{itemize}
\item Deberá funcionar en cualquier sistema \textbf{GNU/Linux} con los
  requisitos anteriormente indicados.
\item Deberá limitarse el número de dependencias, así como facilitar al máximo
  la instalación de las que resultasen imprescindibles.
\item El uso del teclado quedará en segundo plano, haciendo posible utilizar la
  aplicación completamente con el ratón.
\item Al tratarse de un público objetivo juvenil, la aplicación deberá ser
  dinámica, intuitiva y fácil de usar, y la apariencia debe ser agradable.
\item Se evitará el uso de constantes y recursos dentro del código de la
  aplicación, utilizando como alternativa ficheros para representar las
  lecciones y las canciones.
\end{itemize}

\section{Modelo de casos de uso}

A la hora de modelar los casos de uso del sistema, hemos optado por utilizar
notación \textit{UML}, siguiendo los siguientes pasos:
\begin{itemize}
\item Identificación de los usuarios del sistema y sus roles.
\item Para cada rol, determinar las formas de interactuar con el sistema.
\item Creación de casos de uso para los objetivos que debe cumplir la aplicación.
\item Modularización de los casos de usos mediante la implementación de
  relaciones de inclusión o extensión.
\end{itemize}

\subsection{Diagrama de casos de uso}

\begin{figure}[h!]
  \centering
  \includegraphics[width=0.81\textwidth]{4_analisis/imagen_diagrama_de_casos_de_uso}
  \caption{Diagrama de casos de uso}
\end{figure}


\subsection{Descripción de los casos de uso}

\subsubsection{Caso de uso: inicio del juego}
\begin{description}
\item [Descripción] Se muestran los créditos del juego, la pantalla de
  presentación, y finalmente el menú principal, desde donde se accederá al
  resto de secciones del juego.
\item [Actores] \jugador.
\item [Precondiciones] Ninguna.
\item [Postcondiciones] Ninguna.
\item [Escenario principal] $\quad$
  \begin{enumerate}
  \item El \jugador\ inicia la aplicación.
  \item El \sistema\ inicializa el subsistema gráfico.
  \item El \sistema\ muestra la pantalla de créditos y la pantalla de
    presentación de la aplicación.
  \item El \sistema\ muestra el menú principal en la pantalla.
  \item El \jugador\ selecciona la opción \textit{Canciones}.
  \item El \sistema\ accede a la pantalla de \textit{Selección de canción}.
  \end{enumerate}
\item[Extensiones --- flujo alternativo] $\quad$
  \begin{description}
  \item [*a] El \jugador\ cierra la ventana.
    \begin{enumerate}
    \item El \sistema\ libera los recursos y sale de la aplicación.
    \end{enumerate}
  \item [4a] El \jugador\ selecciona la opción \textit{Analizador de Notas}.
    \begin{enumerate}
    \item El \sistema\ accede a la pantalla del analizador de notas.
    \end{enumerate}

  \item[4b] El \jugador\ selecciona la opción \textit{Lecciones}.
    \begin{enumerate}
    \item El \sistema\ accede a la pantalla de \textit{Selección de lecciones}.
    \end{enumerate}
  \item[4c] El \jugador\ selecciona la opción \textit{Calibrar micrófono}.
    \begin{enumerate}
    \item El \sistema\ accede a la pantalla de calibración de micrófono.
    \end{enumerate}
  \item [4d] El \jugador\ selecciona la opción Salir.
    \begin{enumerate}
    \item El \sistema\ libera los recursos y sale de la aplicación.\\
    \end{enumerate}
  \end{description}  
\end{description}


\subsubsection{Caso de uso: selección de canción}

\begin{description}
\item [Descripción] Al \jugador\ se le muestra una lista de las canciones
  detectadas, y éste debe elegir entre ellas la que desea interpretar, o volver
  al menú princial.
\item [Actores] \jugador.
\item [Precondiciones] Ninguna.
\item [Postcondiciones] Una canción queda seleccionada.
\item [Escenario principal] $\quad$
  \begin{enumerate}
  \item El \jugador\ accede, desde el menú principal, al panel de selección de canciones.
  \item El \sistema\ busca las canciones dadas de alta en el juego y muestra un menú con las mismas.
  \item El \jugador\ navega entre las canciones listadas y selecciona una de ellas, pulsando finalmente el botón \textit{Ok}.
  \item El \sistema\ carga la canción y pasa a la pantalla de interpretación de canciones.
  \end{enumerate}
\item[Extensiones --- flujo alternativo] $\quad$
  \begin{description}
  \item [*a] El \jugador\ cierra la ventana.
    \begin{enumerate}
    \item El \sistema\ libera los recursos y sale de la aplicación.
    \end{enumerate}
  \item [3a] El \jugador\ selecciona la opción \textit{Volver}.
    \begin{enumerate}
    \item El \sistema\ muestra la animación de cierre y vuelve al menú principal.
    \end{enumerate}
  \end{description}  
\end{description}


\subsubsection{Caso de uso: interpretación de canción}

\begin{description}
\item [Descripción] Tras haber elegido la canción a interpretar, se muestra una
  partitura con las notas que el \jugador\ deberá tocar para conseguir la puntuación deseada.
\item [Actores] \jugador.
\item [Precondiciones] Se ha elegido una canción.
\item [Postcondiciones] Se completa la interpretación de la canción, obteniendo una calificación
\item [Escenario principal] $\quad$
  \begin{enumerate}
  \item El \sistema\ carga la canción, leyendo las notas, y muestra en pantalla,
    mediante animaciones, el marcador de puntos y el pentagrama.
  \item El \sistema\ comienza a mostrar notas en el pentagrama, que van
    deslizándose hacia el lado izquierdo, en el que se encuentra la aguja de
    reproducción, e inicia el análisis del sonido.
  \item El \jugador\, al llegar la nota a la aguja de reproducción, toca la
    flauta con la altura y la duración correcta, de forma que el micrófono sea
    capaz de captar el sonido.
  \item El \sistema\ analiza el sonido que captura el micrófono y detecta la nota que toca el usuario.
  \item El \sistema\ determina que la nota es la correcta y suma los puntos correspondientes.
  \item Mientras existan más notas, se vuelve al punto 2.
  \item El \sistema\ determina que no hay más notas que mostrar, e inicia las
    animaciones para ocultar los elementos en pantalla.
  \item El \sistema\ pasa a la sección de \textit{Resultados de la interpretación}.
  \end{enumerate}
\item[Extensiones --- flujo alternativo] $\quad$
  \begin{description}

  \item [*a] El \jugador\ cierra la ventana.
    \begin{enumerate}
    \item El \sistema\ libera los recursos y sale de la aplicación.
    \end{enumerate}

  \item[*b] El \jugador\ pulsa la tecla \texttt{escape}.
    \begin{enumerate}
    \item El \sistema\ vuelve a la pantalla de \textit{selección de canción}.
    \end{enumerate}

  \item [3a] El \jugador\ toca el instrumento con intensidad insuficiente o nula
    y el sonido no llega al sistema.
    \begin{enumerate}
    \item El \sistema\ representa esta inconsistencia como un silencio.
    \end{enumerate}

  \item [4a] El \sistema\ no es capaz de determinar fehacientemente la nota que
    toca el usuario.
    \begin{enumerate}
    \item El \sistema\ representa esta inconsistencia como un silencio.
    \end{enumerate}

  \item[5a] El \sistema\ determina que la nota tocada por el usuario no es la
    que corresponde a la partitura.
    \begin{enumerate}
    \item El \sistema\ ignora esta situación y no suma los puntos al marcador.
    \end{enumerate}

  \end{description}  
\end{description}

\subsubsection{Caso de uso: resultados de la interpretación}

\begin{description}
\item [Descripción] Después de interpretar las notas de la partitura, se
  muestran los datos obtenidos del análisis de las notas tocadas por el \jugador.
\item [Actores] \jugador.
\item [Precondiciones] Se ha elegido e interpretado una canción.
\item [Postcondiciones] Se completa la partida actual.
\item [Escenario principal] $\quad$
  \begin{enumerate}
  \item El \sistema\ compara la puntuación conseguida con la máxima puntuación
    obtenible, y genera un porcentaje de aciertos.
  \item El \sistema\ muestra, mediante animaciones, un mensaje con información
    sobre la canción y sobre la interpretación del \jugador\, representada
    mediante un porcentaje de aciertos.
  \item El \sistema\ muestra un mensaje variable en función del número de
    aciertos conseguido.
  \item El \jugador\ revisa su puntuación y pulsa el botón \textit{volver} para
    ir de vuelta al menú de \textit{selección de canción}.
  \end{enumerate}
\item[Extensiones --- flujo alternativo] $\quad$
  \begin{description}

  \item [*a] El \jugador\ cierra la ventana.
    \begin{enumerate}
    \item El \sistema\ libera los recursos y sale de la aplicación.
    \end{enumerate}

  \item[4a] El \jugador\ pulsa la tecla \texttt{escape}.
    \begin{enumerate}
    \item El \sistema\ vuelve a la pantalla de \textit{selección de canción}.
    \end{enumerate}

  \end{description}  
\end{description}


\subsubsection{Caso de uso: analizador de notas}

\begin{description}
\item [Descripción] El \jugador\ elige la opción \textit{analizador de notas} en
  el menú principal y es llevado a esta sección, en la que el sistema
  representará gráficamente la nota que esté tocando con la flauta en cada
  instante, sin otra interacción
\item [Actores] \jugador.
\item [Precondiciones] Ninguna.
\item [Postcondiciones] Ninguna
\item [Escenario principal] $\quad$
  \begin{enumerate}
  \item El \jugador\ accede, desde el menú principal, al panel del analizador de notas.
  \item El \sistema\ muestra, mediante animaciones, la pantalla de la sección,
    representada mediante una fracción de partitura en la que se representará la
    nota que esté tocando el \jugador\ en cada momento.
  \item El \sistema\ inicia el análisis del sonido.
  \item El \jugador\ toca la nota que desee con su flauta, de forma que el
    micrófono sea capaz de captar el sonido.
  \item El \sistema\ analiza el sonido que captura el micrófono y detecta la
    nota que toca el usuario.
  \item El \sistema\ muestra en pantalla la nota, sobre la partitura,
    correspondiente a lo que ha tocado el usuario.
  \item Se repite el flujo desde el punto 4, mientras el \jugador\ no pulse en
    el botón volver.
  \item El \jugador\ pulsa en el botón \textit{volver}.
  \item El \sistema\ inicia las animaciones para ocultar los elementos en pantalla.
  \item El \sistema\ vuelve al menú principal.
  \end{enumerate}
\item[Extensiones --- flujo alternativo] $\quad$
  \begin{description}

  \item [*a] El \jugador\ cierra la ventana.
    \begin{enumerate}
    \item El \sistema\ libera los recursos y sale de la aplicación.
    \end{enumerate}

  \item[*b] El \jugador\ pulsa la tecla \texttt{escape}.
    \begin{enumerate}
    \item El \sistema\ vuelve al menú principal.
    \end{enumerate}

  \item [4a] El \jugador\ toca el instrumento con intensidad insuficiente o nula
    y el sonido no llega al sistema.
    \begin{enumerate}
    \item El \sistema\ representa esta inconsistencia como un silencio.
    \end{enumerate}

  \item [5a] El \sistema\ no es capaz de determinar fehacientemente la nota que
    toca el usuario.
    \begin{enumerate}
    \item El \sistema\ representa esta inconsistencia como un silencio.
    \end{enumerate}
  \end{description}  
\end{description}

\subsubsection{Caso de uso: calibración de micrófono}
\begin{description}
\item [Descripción] El \jugador\ elige la opción \textit{calibrar micrófono} en
  el menú principal y es llevado a esta sección, en la que el \sistema\ calibrará
  el micrófono de forma que sea posible aislar el sonido de la flauta del ruido
  ambiental.
\item [Actores] \jugador.
\item [Precondiciones] Ninguna.
\item [Postcondiciones] El \sistema\ obtiene un valor umbral con el que
  discernir entre el sonido del instrumento y el ruido ambiente.
\item [Escenario principal] $\quad$
  \begin{enumerate}
  \item El \jugador\ accede, desde el menú principal, al panel de calibración del micrófono.
  \item El \sistema\ muestra la sección, indicando con un mensaje que el usuario
    debe pulsar la tecla \texttt{escape} para iniciar la calibración.
  \item El \jugador\ pulsa la tecla \texttt{escape} y se mantiene en silencio.
  \item El \sistema\ inicia el análisis del sonido, guardando durante dos
    segundos los valores de ruido que lee del micrófono.
  \item El \sistema\ calcula, a partir de los valores leídos, el umbral de
    ruido, y muestra un mensaje informando del final del proceso.
  \item El \jugador\ pulsa la tecla \texttt{escape} y el \sistema\ vuelve al
    menú principal.
  \end{enumerate}
\item[Extensiones --- flujo alternativo] $\quad$
  \begin{description}

  \item [*a] El \jugador\ cierra la ventana.
    \begin{enumerate}
    \item El \sistema\ libera los recursos y sale de la aplicación.
    \end{enumerate}

  \item[*b] El \jugador\ pulsa la tecla \texttt{escape}.
    \begin{enumerate}
    \item El \sistema\ cancela la calibración y vuelve al menú principal.
    \end{enumerate}

  \item [5a] El \sistema\ encuentra valores inválidos al leer el ruido
    ambiental.
    \begin{enumerate}
    \item El \sistema\ informa al usuario del fallo del proceso de calibración.
    \item El \jugador\ pulsa la tecla \texttt{escape} y el \sistema\ vuelve al
      menú principal.
    \end{enumerate}
  \end{description}  
\end{description}

\subsubsection{Caso de uso: selección de lecciones}
\begin{description}
\item [Descripción] El \jugador\ elige la opción \textit{lecciones} en el menú
  principal y es llevado a esta sección, en la que el \sistema\ mostrará una
  lista de lecciones cargadas, entre las que el usuario deberá elegir.
\item [Actores] \jugador.
\item [Precondiciones] Ninguna.
\item [Postcondiciones] Se ha elegido una lección
\item [Escenario principal] $\quad$
  \begin{enumerate}
  \item El \jugador\ accede, desde el menú principal, al panel de selección de lecciones.
  \item El \sistema\ carga la lista de secciones y muestra, mediante
    animaciones, el panel, preseleccionando por defecto la primera lección.
  \item El \jugador\ utiliza los botones de la sección para elegir una de las
    lecciones, y activarla pulsando \textit{comenzar lección}.
  \item El \sistema\ oculta de forma animada el panel de selección de lecciones.
  \item El \sistema\ lee el fichero \texttt{xml} asociado a la lección indicada,
    cargando los elementos que la componen y las animaciones que se ejecutarán.
  \item El \sistema\ ejecuta las animaciones correspondientes a los elementos
    multimedia de la lección.
  \end{enumerate}
\item[Extensiones --- flujo alternativo] $\quad$
  \begin{description}

  \item [*a] El \jugador\ cierra la ventana.
    \begin{enumerate}
    \item El \sistema\ libera los recursos y sale de la aplicación.
    \end{enumerate}

  \item[2a] El \sistema\ detecta que una de las lecciones leídas no está
    correctamente construída.
    \begin{enumerate}
    \item El \sistema\ informa del error en el \textit{log} del programa y omite
      la carga de esa lección.
    \end{enumerate}

  \item[3a] El \jugador\ pulsa la tecla \texttt{escape}.
    \begin{enumerate}
    \item El \sistema\ vuelve al menú principal.
    \end{enumerate}

  \item[6a] El \jugador\ pulsa la tecla \texttt{escape}.
    \begin{enumerate}
    \item El \sistema\ vuelve al menú de selección de lecciones.
    \end{enumerate}

  \end{description}  
\end{description}

\section{Modelo conceptual de datos}

El modelo conceptual de datos representa, de forma esquemática, las clases que
modelan el sistema y las relaciones que existen entre ellas, además de una
pequeña introducción a su utilidad. 

\begin{description}
\item[Juego] Clase de control general. Gestiona el flujo de ejecución principal,
  así como de la gestión de estados, que permite pasar de una sección a otra del
  juego.
\item[Estado] Clase base para los diferentes estados del juego. Las clases
  correspondientes a las secciones se basarán en esta clase para interactuar con
  el gestor de estados y poder pasar de una parte del juego a otra.
\item[EstadoMenú] Representa el estado de juego para el menú principal, desde el
  que se accede al resto de opciones del juego. 
\item[EstadoAnalizador] Representa el estado del analizador básico de
  notas. Contendrá los elementos necesarios para iniciar el análisis del audio,
  así como los elementos de la interfaz.
\item[EstadoCalibrarMicro] Representa el estado en el que se calibra el
  micrófono. Al igual que la clase \textit{EstadoAnalizador}, deberá ser capaz
  de acceder al sistema de audio para poder leer el volumen ambiente y así
  calibrar el micrófono.
\item[EstadoImagenFija] Modela una imagen fija a modo de pantalla de créditos,
  de forma que sea sencillo añadir imágenes al inicio del juego, como firmas de
  desarrolladores, logotipos de patrocinadores, etcétera.
\item[EstadoMenúCanciones] Comprende el menú de selección de canciones, que se
  encargará de leer los ficheros de canciones disponibles. Además, también se
  encargará de lanzar las canciones en forma de estados secundarios.
\item[EstadoCanción] Corresponde a la canción que se va a interpretar, lanzada desde
  el estado \textit{EstadoMenuCanciones}.
\item[EstadoMenúLecciones] Corresponde al menú de elección de lecciones, que
  leerá y listará los ficheros de lección disponibles, y se encargará de lanzar
  la lección elegida.
\item[EstadoLección] Corresponde a la canción elegida desde el menú
  \textit{EstadoMenúLecciones}.
\item[Analizador] Controla la gestión del subsistema de audio y el análisis de
  la entrada. Deberá proporcionar información sobre el volumen de la entrada
  (para la calibración del micrófono) así como de la nota detectada en cada
  instante.
\item[Animación] Se encargará de facilitar la creación de animaciones en forma
  de interpolación de valores, válidas para cambios de posición, opacidad,
  etcétera.
\item[Elemento] Esta clase de ayuda facilitará la carga y dibujado de elementos
  para la interfaz, además de servir de capa de abstracción para las
  animaciones.
\item[ElementoImagen] Especialización de la clase \textit{Elemento} para
  imágenes.
\item[ElementoTexto] Especialización de la clase \textit{Elemento} para textos.
\item[ElementoCombinado] Especialización de la clase \textit{Elemento} que
  combina imagen y texto, a usar en casos como los botones del menú.
\item[SistemaPartículas] Representa un sistema de partículas simple, para
  generar efectos de destellos y fuegos artificiales.
\item[Nota] Simboliza cada una de las notas cargadas que componen una canción.
\end{description}

\begin{figure}[htp!]
  \centering
  \includegraphics[angle=90]{4_analisis/imagen_diagrama_clases_conceptuales}
  \caption{Diagrama de clases conceptuales}
\end{figure}

\pagebreak


\section{Modelo de comportamiento del sistema}
En esta sección vamos a especificar cómo se comporta el sistema en forma de dos
elementos fundamentales.
\begin{itemize}
\item En primer lugar, los \textbf{diagramas de secuencia} mostrarán el flujo de
  eventos entre los actores que participan en la aplicación.
\item En segundo lugar, los \textbf{contratos de las operaciones} detallarán las
  condiciones y efectos que tendrán lugar al ejecutarse las operaciones en el
  sistema.
\end{itemize}

\begin{nota}
  No se han reflejado, por triviales, los escenarios alternativos en los que el
  usuario cierra la ventana, correspondientes a los flujos \textit{*a} definidos
  en la sección anterior.
\end{nota}

\subsection{Caso de uso: inicio del juego}

\subsubsection{Escenario principal}

\begin{figure}[h!]
  \centering
  \includegraphics[trim=0cm 12cm 0cm 0cm, clip=true, width=0.5\textwidth]{4_analisis/diagsec_caso1_esc1}
  \caption{Diagrama de secuencia, incio del juego, escenario principal}
\end{figure}

\begin{description}
\item[Operación] IniciarJuego()
\item[Actores] \jugador\, \sistema\
\item[Responsabilidades] Cargar y lanzar la aplicación, mostrar los títulos de
  crédito y el menú principal.
\item[Precondiciones] Ninguna.
\item[Postcondiciones] $\quad$

  \begin{itemize}
  \item Se crea una instancia de la clase \textit{Juego}, que gestiona la
    creación y destrucción de los estados.
  \item Se crean y posteriormente destruyen dos estados
    \textit{EstadoImagenFija} para mostrar los títulos de crédito.
  \item Se crea y permanece un estado \textit{EstadoMenú}, que representa el
    menú principal de la aplicación.
  \end{itemize}

\end{description}

\begin{description}
\item[Operación] SeleccionarSeccionCanciones()
\item[Actores] \jugador\, \sistema\
\item[Responsabilidades] Esconder el menú principal y cargar el menú de
  selección de canciones.
\item[Precondiciones] $\quad$

  \begin{itemize}
  \item El estado actual es una instancia de \textit{EstadoMenú}.
  \end{itemize}

\item[Postcondiciones] Se destruye el estado \textit{EstadoMenú} y se carga
  \textit{EstadoMenúCanciones}.
\end{description}

\subsubsection{Escenario alternativo 4a}
\begin{figure}[h!]
  \centering
  \includegraphics[trim=0cm 12cm 0cm 0cm, clip=true, width=0.5\textwidth]{4_analisis/diagsec_caso1_esc2}
  \caption{Diagrama de secuencia, incio del juego, escenario alternativo 4a}
\end{figure}

\begin{description}
\item[Operación] SeleccionarSeccionAnalizador()
\item[Actores] \jugador\, \sistema\
\item[Responsabilidades] Esconder el menú principal y cargar la sección de
  análisis de notas.
\item[Precondiciones] $\quad$
  \begin{itemize}
  \item El estado actual es una instancia de \textit{EstadoMenú}.
  \end{itemize}
\item[Postcondiciones] Se destruye el estado \textit{EstadoMenú} y se carga
  \textit{EstadoAnalizador}.
\end{description}

\subsubsection{Escenario alternativo 4b}
\begin{figure}[h!]
  \centering
  \includegraphics[trim=0cm 12cm 0cm 0cm, clip=true, width=0.5\textwidth]{4_analisis/diagsec_caso1_esc3}
  \caption{Diagrama de secuencia, incio del juego, escenario alternativo 4b}
\end{figure}

\begin{description}
\item[Operación] SeleccionarSeccionLecciones()
\item[Actores] \jugador\, \sistema\
\item[Responsabilidades] Esconder el menú principal y cargar la menú de
  selección de lecciones.
\item[Precondiciones] $\quad$
  \begin{itemize}
  \item El estado actual es una instancia de \textit{EstadoMenú}.
  \end{itemize}
\item[Postcondiciones] Se destruye el estado \textit{EstadoMenú} y se carga
  \textit{EstadoMenúLecciones}.
\end{description}

\subsubsection{Escenario alternativo 4c}
\begin{figure}[h!]
  \centering
  \includegraphics[trim=0cm 12cm 0cm 0cm, clip=true, width=0.5\textwidth]{4_analisis/diagsec_caso1_esc4}
  \caption{Diagrama de secuencia, incio del juego, escenario alternativo 4c}
\end{figure}

\begin{description}
\item[Operación] SeleccionarSeccionCalibracion()
\item[Actores] \jugador\, \sistema\
\item[Responsabilidades] Esconder el menú principal y cargar la sección de
  calibración de micrófono.
\item[Precondiciones] $\quad$
  \begin{itemize}
  \item El estado actual es una instancia de \textit{EstadoMenú}.
  \end{itemize}
\item[Postcondiciones] Se destruye el estado \textit{EstadoMenú} y se carga
  \textit{EstadoCalibrarMicro}.
\end{description}

\subsubsection{Escenario alternativo 4d}
\begin{figure}[h!]
  \centering
  \includegraphics[trim=0cm 12cm 0cm 0cm, clip=true, width=0.5\textwidth]{4_analisis/diagsec_caso1_esc5}
  \caption{Diagrama de secuencia, incio del juego, escenario alternativo 4d}
\end{figure}

\begin{description}
\item[Operación] SeleccionarSalir()
\item[Actores] \jugador\, \sistema\
\item[Responsabilidades] Esconder el menú principal, descargar los recursos y
  cerrar la aplicación.
\item[Precondiciones] $\quad$
  \begin{itemize}
  \item El estado actual es una instancia de \textit{EstadoMenú}.
  \end{itemize}
\item[Postcondiciones] Se destruye el estado \textit{EstadoMenú}, se destruye la
  instancia de la clase \textit{Juego} y termina la ejecución de la aplicación.
\end{description}

\subsection{Selección de canción}

\subsubsection{Escenario principal}
\begin{figure}[h!]
  \centering
  \includegraphics[trim=0cm 12cm 0cm 0cm, clip=true, width=0.5\textwidth]{4_analisis/diagsec_caso2_esc1}
  \caption{Diagrama de secuencia, selección de canción, escenario principal}
\end{figure}

\begin{description}
\item[Operación] ListarCanciones()
\item[Actores] \jugador\, \sistema\
\item[Responsabilidades] Cargar y mostrar la lista de canciones cargadas en el
  sistema.
\item[Precondiciones] Se ordenó la carga del estado \textit{EstadoMenuCanción}
\item[Postcondiciones] $\quad$
  \begin{itemize}
  \item El estado actual es una instancia de \textit{EstadoMenuCanción}.
  \item Se ha cargado la lista de canciones y se muestra en pantalla.
  \end{itemize}
\end{description}

\begin{description}
\item[Operación] ElegirCanción()
\item[Actores] \jugador\, \sistema\
\item[Responsabilidades] Cargar la canción que el usuario ha elegido para
  interpretar.
\item[Precondiciones] Existe una lista de canciones cargada de entre las que el
  usuario ha elegido una.
\item[Postcondiciones] $\quad$
  \begin{itemize}
  \item Se carga la canción indicada.
  \item Se oculta la lista de canciones.
  \item Se pasa a un sub-estado de interpretación de canción.
  \end{itemize}
\end{description}
\subsubsection{Escenario alternativo 3a}
\begin{figure}[h!]
  \centering
  \includegraphics[trim=0cm 12cm 0cm 0cm, clip=true, width=0.5\textwidth]{4_analisis/diagsec_caso2_esc2}
  \caption{Diagrama de secuencia, selección de canción, escenario alternativo 3a}
\end{figure}
\begin{description}
\item[Operación] ElegirVolver()
\item[Actores] \jugador\, \sistema\
\item[Responsabilidades] Descargar la sección actual y volver al menú anterior.
\item[Precondiciones] El estado actual es una instancia de \textit{EstadoMenuCanción}.
\item[Postcondiciones] $\quad$
  \begin{itemize}
  \item El estado instancia de \textit{EstadoMenuCanción} queda descargado.
  \item Se carga y se muestra \textit{EstadoMenú}.
  \end{itemize}
\end{description}

\subsection{Interpretación de canción}
\begin{nota}
  No se reflejan los escenarios alternativos al estar englobados en la operación
  \textit{InteractuarConFlauta}.
\end{nota}

\subsubsection{Escenario principal}
\begin{figure}[h!]
  \centering
  \includegraphics[trim=0cm 8cm 0cm 0cm, clip=true, width=0.5\textwidth]{4_analisis/diagsec_caso3}
  \caption{Diagrama de secuencia, interpretación de canción, escenario principal}
\end{figure}

\begin{description}
\item[Operación] IniciarInterpretación()
\item[Actores] \jugador\, \sistema\
\item[Responsabilidades] Parsear el fichero de canción, cargar la interfaz y
  comenzar la interpretación.
\item[Precondiciones] El usuario ha elegido una canción en el estado anterior.
\item[Postcondiciones] $\quad$
  \begin{itemize}
  \item Se muestra la interfaz de interpretación de canción.
  \item El fichero de canción queda cargado e interpretado, instanciando los
    elementos de la clase \textit{Nota} que sean necesarios.
  \item Comienza la interpretación
  \end{itemize}
\end{description}

\begin{description}
\item[Operación] InteractuarConFlauta()
\item[Actores] \jugador\, \sistema\
\item[Responsabilidades] El \jugador\ interactúa con el sistema mediante la
  flauta a través del micrófono, y el \sistema\ analiza los datos y muestra una
  respuesta en pantalla.
\item[Precondiciones] $\quad$
  \begin{itemize}
  \item La interpretación ha comenzado.
  \item El micrófono está correctamente configurado.
  \end{itemize}
\item[Postcondiciones] $\quad$
  \begin{itemize}
  \item El sistema captura y analiza los datos de audio.
  \item Según el análisis, el sistema responde de una forma u otra (según los
    escenarios alternativos 3a, 4a y 5a del caso de uso \textit{interpretación
      de canción}).
  \end{itemize}
\end{description}

\begin{description}
\item[Operación] FinalizarInterpretación()
\item[Actores] \jugador\, \sistema\
\item[Responsabilidades] Descargar la pantalla de interpretación, descargar la
  canción y finalizar la interpretación.
\item[Precondiciones] Todas las notas se han interpretado.
\item[Postcondiciones] $\quad$
  \begin{itemize}
  \item Se descarga la \textit{Canción} actual.
  \item Se ocultan los elementos de la interfaz de interpretación.
  \item Se lanza la sección de puntuación.
  \end{itemize}
\end{description}


\subsection{Resultados de interpretación}

\subsubsection{Escenario principal}
\begin{figure}[h!]
  \centering
  \includegraphics[trim=0cm 12cm 0cm 0cm, clip=true, width=0.5\textwidth]{4_analisis/diagsec_caso4}
  \caption{Diagrama de secuencia, resultados de interpretación, escenario principal}
\end{figure}

\begin{description}
\item[Operación] MostrarResultados()
\item[Actores] \jugador\, \sistema\
\item[Responsabilidades] Interpretar los resultados de la interpretación y
  mostrar los resultados en pantalla.
\item[Precondiciones] El \jugador\ ha concluído satisfactoriamente una
  interpretación completa de una canción, obteniendo una suma de puntos $X$.
\item[Postcondiciones] Mostrar en pantalla los resultados en forma de porcentaje
  de aciertos, y un mensaje según aquél.
\end{description}

\begin{description}
\item[Operación] ElegirVolver()
\item[Actores] \jugador\, \sistema\
\item[Responsabilidades] Descargar la sección actual y volver al menú anterior.
\item[Precondiciones] $\quad$
  \begin{itemize}
  \item La aplicación se encuentra en la pantalla de muestra de resultados.
  \item El usuario ha pulsado la tecla \texttt{escape} o el botón \textit{volver}.
  \end{itemize}
\item[Postcondiciones] $\quad$
  \begin{itemize}
  \item Se descargan todos los datos referentes a la canción actual.
  \item Se carga y se muestra \textit{EstadoMenúCanciones}.
  \end{itemize}
\end{description}

\subsection{Analizador de notas}

\begin{nota}
  No se reflejan los escenarios alternativos al estar englobados en la operación
  \textit{InteractuarConFlauta}.
\end{nota}

\subsubsection{Escenario principal}
\begin{figure}[h!]
  \centering
  \includegraphics[trim=0cm 8cm 0cm 0cm, clip=true, width=0.5\textwidth]{4_analisis/diagsec_caso5}
  \caption{Diagrama de secuencia, interpretación de canción, escenario principal}
\end{figure}

\begin{description}
\item[Operación] IniciarAnálisis
\item[Actores] \jugador\, \sistema\
\item[Responsabilidades] Cargar la interfaz e iniciar el análisis de notas.
\item[Precondiciones] El usuario eligió la sección \textit{Analizador de notas}
  en el menú principal.
\item[Postcondiciones] $\quad$
  \begin{itemize}
  \item Aparece la interfaz del analizador de notas.
  \item Se inicia el análisis de notas
  \end{itemize}
\end{description}

\begin{description}
\item[Operación] InteractuarConFlauta
\item[Actores] \jugador\, \sistema\
\item[Responsabilidades] El \jugador\ toca notas en la flauta y el \sistema\
  captura y reconoce el audio, indicando la nota tocada en pantalla.

\item[Precondiciones] Se ha iniciado el análisis.

\item[Postcondiciones] $\quad$
  \begin{itemize}
  \item El \sistema\ recoge y analiza el sonido que emite la flauta del \jugador.
  \item El \sistema\ representa en pantalla la nota identificada, o no muestra
    nada en caso de identificación defectuosa.
  \end{itemize}
\end{description}

% \begin{description}
% \item[Operación] ElegirVolver()
% \item[Actores] \jugador\, \sistema\
% \item[Responsabilidades] Descargar la sección actual y volver al menú anterior.
% \item[Precondiciones] $\quad$
%   \begin{itemize}
%   \item El estado actual es una instancia de \textit{EstadoAnalizador}.
%   \item Hay un análisis en curso.  
%   \end{itemize}
  
% \item[Postcondiciones] $\quad$
%   \begin{itemize}
%   \item Concluye el análisis actual.
%   \item Se destruye el estado.
%   \item Se carga y se muestra \textit{EstadoMenú}.
%   \end{itemize}
% \end{description}

\subsection{Calibración de micrófono}

\subsubsection{Escenario principal}
\begin{figure}[h!]
  \centering
  \includegraphics[trim=0cm 8cm 0cm 0cm, clip=true, width=0.5\textwidth]{4_analisis/diagsec_caso6_esc1}
  \caption{Diagrama de secuencia, calibración de micrófono, escenario principal}
\end{figure}

\begin{description}
\item[Operación] CargarCalibración()
\item[Actores] \jugador\, \sistema\
\item[Responsabilidades] Cargar la sección y preparar el sistema para comenzar
  la calibración del micrófono.
\item[Precondiciones] El usuario ha elegido en el menú principal la opción
  \textit{Calibrar micrófono}.
\item[Postcondiciones] La sección está cargada y la calibración lista para
  iniciarse.
\end{description}

\begin{description}
\item[Operación] CalibrarMicrófono()
\item[Actores] \jugador\, \sistema\
\item[Responsabilidades] Llevar a cabo la calibración correcta del micrófono.
\item[Precondiciones] El usuario ha lanzado la calibración del micrófono.
\item[Postcondiciones] $\quad$
  \begin{itemize}
  \item Se cierra el sistema de sonido.
  \item Se obtiene un valor umbral de ruido ambiente, fruto de una calibración
    exitosa.
  \end{itemize}
\end{description}

\begin{description}
\item[Operación] ElegirVolver()
\item[Actores] \jugador\, \sistema\
\item[Responsabilidades] Descargar la sección actual y volver al menú anterior.
\item[Precondiciones] $\quad$
  \begin{itemize}
  \item La calibración ha concluído exitosamente.
  \item El usuario ha pulsado la tecla \texttt{escape} o el botón \textit{volver}.
  \end{itemize}
\item[Postcondiciones] $\quad$
  \begin{itemize}
  \item Se descarga la sección actual.
  \item Se carga y se muestra \textit{EstadoMenúCanciones}.
  \end{itemize}
\end{description}

\subsubsection{Escenario alternativo 5a}
\begin{figure}[h!]
  \centering
  \includegraphics[trim=0cm 8cm 0cm 0cm, clip=true, width=0.5\textwidth]{4_analisis/diagsec_caso6_esc2}
  \caption{Diagrama de secuencia, calibración de micrófono, escenario principal}
\end{figure}
\begin{description}
\item[Operación] CalibrarMicrófono()
\item[Actores] \jugador\, \sistema\
\item[Responsabilidades] Llevar a cabo la calibración correcta del micrófono.
\item[Precondiciones] El usuario ha lanzado la calibración del micrófono.
\item[Postcondiciones] $\quad$
  \begin{itemize}
  \item Se cierra el sistema de sonido.
  \item La calibración ha fallado.
  \item No se obtiene valor umbral de ruido ambiental.
  \end{itemize}
\end{description}

\subsection{Selección de lecciones}

\subsubsection{Escenario principal}
\begin{figure}[h!]
  \centering
  \includegraphics[trim=0cm 4cm 0cm 0cm, clip=true, width=0.5\textwidth]{4_analisis/diagsec_caso7_esc1}
  \caption{Diagrama de secuencia, selección delecciones, escenario principal}
\end{figure}


\begin{description}
\item[Operación] ListarLecciones()
\item[Actores] \jugador\, \sistema\
\item[Responsabilidades] Mostrar el menú de selección de lecciones y listar
  todas las lecciones disponibles.
\item[Precondiciones] El usuario eligió la opción \textit{Lecciones} en el menú
  principal.
\item[Postcondiciones] $\quad$
  \begin{itemize}
  \item Se muestra la interfaz del menú de selección de lecciones.
  \item Se listan las lecciones cargadas en el sistema
  \end{itemize}
\end{description}

\begin{description}
\item[Operación] ElegirLección()
\item[Actores] \jugador\, \sistema\
\item[Responsabilidades] Cargar y mostrar la lección elegida por el usuario.
\item[Precondiciones] El menú de selección de lecciones está cargado y el
  usuario ha elegido una de las lecciones.
\item[Postcondiciones]$\quad$
  \begin{itemize}
  \item Se oculta el menú de selección de lecciones.
  \item Se interpreta el fichero de lección elegido.
  \item Se muestran los elementos multimedia pertenecientes a la lección
    elegida.
  \end{itemize}
\end{description}

\begin{description}
\item[Operación] VolverMenuLecciones()
\item[Actores] \jugador\, \sistema\
\item[Responsabilidades] Descargar la lección actual y volver al menú anterior.
\item[Precondiciones] $\quad$
  \begin{itemize}
  \item El usuario ha terminado de ver la lección elegida.
  \item El usuario ha pulsado la tecla \texttt{escape} o el botón \textit{volver}.
  \end{itemize}
\item[Postcondiciones] $\quad$
  \begin{itemize}
  \item Se descarga la lección actual.
  \item Se muestra de nuevo el menú de selección de lecciones.
  \end{itemize}
\end{description}

\begin{description}
\item[Operación] VolverMenuPrincipal()
\item[Actores] \jugador\, \sistema\
\item[Responsabilidades] Descargar el menú de selección de lecciones y volver al
  menú principal.
\item[Precondiciones] El usuario ha pulsado la tecla \texttt{escape} o el botón \textit{volver}.
\item[Postcondiciones] $\quad$
  \begin{itemize}
  \item Se descarga el menú de selección de lecciones.
  \item Se carga y muestra el menú principal
  \end{itemize}
\end{description}

\subsubsection{Escenario alternativo}
\begin{figure}[h!]
  \centering
  \includegraphics[trim=0cm 12cm 0cm 0cm, clip=true, width=0.5\textwidth]{4_analisis/diagsec_caso7_esc2}
  \caption{Diagrama de secuencia, selección de lecciones, escenario alternativo}
\end{figure}

\begin{description}
\item[Operación] ListarLecciones()
\item[Actores] \jugador\, \sistema\
\item[Responsabilidades] Mostrar el menú de selección de lecciones y listar
  todas las lecciones disponibles.
\item[Precondiciones] El usuario eligió la opción \textit{Lecciones} en el menú
  principal.
\item[Postcondiciones] $\quad$
  \begin{itemize}
  \item Se muestra la interfaz del menú de selección de lecciones.
  \item Se listan las lecciones cargadas en el sistema
  \end{itemize}
\end{description}

\begin{description}
\item[Operación] VolverMenuPrincipal()
\item[Actores] \jugador\, \sistema\
\item[Responsabilidades] Descargar el menú de selección de lecciones y volver al
  menú principal.
\item[Precondiciones] El usuario ha pulsado la tecla \texttt{escape} o el botón \textit{volver}.
\item[Postcondiciones] $\quad$
  \begin{itemize}
  \item Se descarga el menú de selección de lecciones.
  \item Se carga y muestra el menú principal
  \end{itemize}
\end{description}

%%% Local Variables: 
%%% mode: latex
%%% TeX-master: "../memoria"
%%% End: 

%%% Local Variables: 
%%% mode: latex
%%% TeX-master: "../memoria"
%%% End: 
