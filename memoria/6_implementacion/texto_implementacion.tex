Como complemento a la lectura de este capítulo se recomienda tener una copia
local del repositorio del proyecto, disponible para su libre descarga desde la
forja oficial~\cite{forja}.

\section{Entorno de construcción}

En esta sección se detalla el marco tecnológico utilizado para al construcción
del sistema, basándonos en lo presentado en la sección~\ref{sec:alternativas-solucion},~\nameref{sec:alternativas-solucion}.

\subsection{Aplicación web}

El entorno tecnológico de la aplicación web ha resultado ser muy variado y rico
en tecnologías en total vigencia en la actualidad. 

\subsubsection{Herramientas de diseño y desarrollo}

Como editor principal se ha utilizado \textbf{Sublime Text
  2}~\cite{sublime-text}, un editor freeware escrito en Python especialmente
dotado para el desarrollo de aplicaciones web. Cuenta con un gran número de
características que lo han hecho destacar en los últimos años, como la habilidad
de seleccionar y editar a la vez varios fragmentos de texto, el rápido motor de
\textit{búsqueda difusa} para encontrar ficheros y cadenas o su capacidad de
extensión a través de paquetes escritos por terceros y fácilmente instalables.

Para ciertas tareas, como la escritura de la presente memoria, se ha hecho uso
del veterano editor \textbf{Emacs}~\cite{emacs}, que sigue siendo la mejor
opción a la hora de editar documentos de \LaTeX. 

Se ha utilizado \textbf{draw.io}~\cite{draw.io} para la generación de diagramas
de casos de uso, secuencia, diagramas de flujo y similares. Se trata de una
herramienta web, actualmente integrable en Google Drive, con un enorme número de
elementos de dibujo y la capacidad de exportar en un gran número de formatos,
entre ellos en formato vectorial \ac{PDF}. Para el diseño visual de las
interfaces de usuario se ha utilizado \textbf{Adobe Photoshop}~\cite{photoshop}.

\subsubsection{Gestión de dependencias}

Para facilitar la gestión de las dependencias del proyecto se han utilizado
\textbf{VirtualEnv}~\cite{virtualenv} y
\textbf{VirtualEnvWrapper}~\cite{virtualenvwrapper}. Estas herramientas permiten
generar \textit{entornos virtuales} para cada proyecto, en los que se instalan
las dependencias necesarias. Estos entornos se activan y desactivan, de forma
que las bibliotecas instaladas en el entorno virtual de un proyecto no son
accesibles desde el entorno del otro proyecto. Esto evita la polución del nivel
general de bibliotecas y facilita el control estricto de las versiones de las
dependencias. 

Junto a virtualenv, la herramienta \textbf{pip}~\cite{pip} permite guardar en un
fichero anexo la lista de dependencias de un proyecto, de forma que sea fácil
reinstalarlas todas si hubiese que repetir la instalación en otro sistema.

\subsubsection{Control de versiones}

Todo el código fuente del proyecto se encuentra alojado en un repositorio
público en \textbf{GitHub}~\cite{forja}, haciendo uso de los planes
gratuitos. GitHub es una forja que utiliza el sistema de control de versiones
\textbf{Git}~\cite{git}. Además del alojamiento de código, GitHub provee
numerosas funcionalidades adicionales, tanto a nivel social (permitiendo a las
forjas tener \textit{followers}, por ejemplo) como a nivel funcional (ofreciendo
sistemas de tickets, estadísticas, etcétera).

El uso de un control de versiones es fundamental por varios motivos. En primer
lugar, sirve como sistema de copia de seguridad. En segundo lugar, permite
deshacer cambios en el código que no funcionen bien, siendo siempre posible
\textit{volver atrás}. Por último, sirve como \textit{cuaderno de bitácora}
improvisado, ya que se guarda el historial de \textit{commits} que el
desarrollador va enviando junto a los mensajes, siendo posible ver en una línea
temporal el progreso del trabajo.

\subsubsection{Lenguaje de programación}

Como se ha comentado en numerosas ocasiones en la presente memoria, el lenguaje
de programación elegido para el desarrollo del proyecto es
\textbf{Python}~\cite{Python}, un lenguaje interpretado de alto nivel
desarrollado por Guido Van Rossum en 1991. Python soporta múltiples paradigmas
de programación, desde la orientación a objetos hasta la programación funcional,
pasando por el clásico estilo imperativo. Su principal uso ha sido como lenguaje
de scripting, pero también tiene su hueco en contextos más amplios como lenguaje
principal.

Su facilidad de aprendizaje, lo simple de su sintaxis (basada en la indentación
para marcar los bloques) y su extensibilidad (sobre todo gracias al uso de
métodos \textit{mágicos} y metaprogramación) han hecho que el lenguaje sea muy
popular y su uso en los últimos años se haya expandido enormemente.

\subsubsection{Framework de desarrollo}

El framework de desarrollo elegido es \textbf{Django}~\cite{Django}, un
proyecto de código abierto nacido en 2005. La meta fundamental de Django es
facilitar la creación de sitios web complejos, haciendo especial hincapié en
mantener un ritmo de desarrollo rápido y evitar la duplicidad de código. Así,
gran parte del potencial de Django es la gran cantidad de funcionalidad que ya
trae integrada, ya sea en el propio núcleo del framework o a través de
\textit{apps} oficiales que se incluyen en la distribución oficial.

Los elementos más destacables de Django son:

\begin{itemize}
\item Un sistema de mapeo objeto-relacional, que facilita enormemente el trabajo
  con elementos de bases de datos a través de instancias de modelos.
\item Un sistema de procesamiento de peticiones a través de vistas.
\item Un despachador de URLs basado en expresiones regulares.
\item Soporte para plantillas.
\item Un sistema de autenticación extensible.
\item Una interfaz de administración que se adapta dinámicamente a cualquier proyecto.
\item Un sistema de comentarios.
\item Soporte integrado contra numerosos tipos de ataques vía web, como
  inyección SQL o cross-site scripting.
\end{itemize}

\subsubsection{Nivel de persistencia}

\textbf{SQLite} es un sistema de gestión de bases de datos relacional de dominio público,
contenido en una pequeña biblioteca. A diferencia de los sistema de gestión de
bases de datos cliente-servidor, el motor de SQLite no es un proceso
independiente con el que el programa principal se comunica. En lugar de eso, la
biblioteca SQLite se enlaza con el programa pasando a ser parte integral del
mismo. El programa utiliza la funcionalidad de SQLite a través de llamadas
simples a subrutinas y funciones. Esto reduce la latencia en el acceso a la base
de datos, debido a que las llamadas a funciones son más eficientes que la
comunicación entre procesos. 

El conjunto de la base de datos (definiciones, tablas, índices, y los propios
datos), son guardados como un sólo fichero estándar en la máquina host. Este
diseño simple se logra bloqueando todo el fichero de base de datos al principio
de cada transacción. Esto no impide que varios procesos o hilos pueden acceder a
la misma base de datos sin problemas.

En general, SQLite no suele ser la mejor opción en sistemas de producción muy
extensos, pero para proyectos de menor envergadura su rendimiento es más que
suficiente.

\subsubsection{Gestión de tareas}

Para la gestión de tareas de forma asíncrona se ha utilizado \textbf{Celery},
una cola de tareas asíncrona enfocada a operaciones en tiempo real. Celery está
escrito en Python, pero se integra fácilmente con cualquier otro sistema que
implemente su interfaz. Las unidades de ejecución básicas, las \textit{tareas},
se ejecutan de forma concurrente en uno o varios \textit{workers}. Las tareas
pueden ejecutarse de forma asíncrona, en segundo plano, o de forma síncrona,
haciendo que el flujo de ejecución espere hasta que el resultado de la tarea
esté disponible.

Celery se comunica mediante mensajes, normalmente utilizando un \textbf{broker}
que media entre los clientes y los workers. Para iniciar una tarea, el cliente
pone un mensaje en la cola del broker. Entonces, el broker entrega el mensaje al
worker de Celery. La opción recomendada, y la que se ha utilizado en SiteUp, es
usar \textbf{RabbitMQ} como broker de mensajes.

RabbitMQ, y la mayoría de brókers de mensajes, implementan el protocolo
\ac{AMQP}, un estándar abierto para enviar mensajes entre aplicaciones y
organizaciones, siempre que éstas sean capaces de interactuar usando las
interfaces del protocolo.

\subsubsection{Dependencias de Django}

En el proyecto se han utilizado un gran número

%%% Local Variables: 
%%% mode: latex
%%% TeX-master: "../memoria"
%%% End: 
